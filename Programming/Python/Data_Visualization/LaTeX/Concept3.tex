% Concept3: Explanatory Visuals
We've examined the data using multiple methods, testing multiple ideas, and have formulated a story the data can tell. Now where do we go? Let's recall the entire data analysis process:

\begin{itemize}
	\item Extract
	\item Clean
	\item Explore
	\item Analyze
	\item Share
\end{itemize} 

We're in the endgame when it comes to the process, so here are some tips for telling a story and finalizing some visualizations.
\\\\

Telling a Story:
\begin{enumerate}
	\item Start with a Question
	\item Use Repetition
	\item Highlight the Answer
	\item Call your Audience to Action
\end{enumerate}

Polishing Plots:
\begin{itemize}
	\item Appropriate Plot Type and Encodings
	\item Good Design Integrity
	\item Labeled Axes, Reasonable Tick Marks
	\item Descriptive Legend and Titles
	\item Accompanying Comments and Text
\end{itemize}

We'll use the pokemon dataset a final time to showcase a polished plot.

\begin{python}
	# We'll rerun the code to create variations for type and specify fairy vs. dragon, with some slight differences
	
	type_cols = ['type_1','type_2']
	non_type_cols = pokemon.columns.difference(type_cols)
	pkmn_types = pokemon.melt(id_vars = non_type_cols, value_vars = type_cols, var_name = 'type_level', value_name = 'type').dropna()
	
	pokemon_sub = pkmn_types.loc[pkmn_types['type'].isin(['fairy','dragon'])]
	
	type_cols = ['type_1','type_2']
	non_type_cols = pokemon.columns.difference(type_cols)
	pkmn_types = pokemon.melt(id_vars = non_type_cols, value_vars = type_cols, 
	var_name = 'type_level', value_name = 'type').dropna()
	
	pokemon_sub = pkmn_types.loc[pkmn_types['type'].isin(['fairy','dragon'])]
	# See Figure 45
\end{python}

\begin{figure}
	\includegraphics[width=\textwidth,height=\textheight,keepaspectratio]{images/figure45.png}
	\caption{Polished Plot.}\label{fig:figure45}
\end{figure}