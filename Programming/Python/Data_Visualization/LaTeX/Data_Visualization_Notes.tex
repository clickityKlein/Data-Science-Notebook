\documentclass{article}
\usepackage{listings}
\usepackage[colorlinks=true,linkcolor=black,anchorcolor=black,citecolor=black,filecolor=black,menucolor=black,runcolor=black,urlcolor=black]{hyperref}
\usepackage{pythonhighlight}
\usepackage{graphicx}
\setlength{\parindent}{0pt}

\begin{document}
	\title{Python: Data Visualization Notes}
	\author{Klein \\ carlj.klein@gmail.com}
	\date{}
	\maketitle
	
\section{Learning Objectives}
What good is a data analysis if the answers or findings can't be conveyed properly? Or perhaps even more detrimental, what good is a dataset if the proper questions or hypotheses can't be formed in the first place? How can we properly tell the story the data is providing? Enter in data visualization! Both explaining the data and exploring the data can be significantly helped through the process of data visualization.
\\\\
Data visualization is two-pronged:
\begin{itemize}
	\item Exploratory Analysis: Helping to understand the data prior to analysis. Searching for relationships and insights.
	\item Explanatory Analysis: Helping to present analysis findings, helping to tell a story with the data. After insights were found.
\end{itemize}


And it's all a part of the entire data analysis process. We can also simplify the data analysis process into 5 steps:
\begin{enumerate}
	\item Extract
	\item Clean: Exploratory
	\item Explore: Exploratory
	\item Analyze: Exploratory OR Explanatory
	\item Share: Explanatory
\end{enumerate}

In this course, we'll be using the matplotlib, seaborn, and Pandas libraries to assist in the data analysis process.
\\\\

Concepts
\begin{itemize}
	\item \nameref{sec:concept1}
	\item \nameref{sec:concept2}
		\begin{itemize}
			\item \nameref{sec:concept2.1}
			\item \nameref{sec:concept2.2}
			\item \nameref{sec:concept2.3}
		\end{itemize}
	\item \nameref{sec:concept3}
	\item \nameref{sec:concept4}
\end{itemize}
	
\section{Design of Visualization}\label{sec:concept1}
% Creating and Saving NumPy ndarrays

Not needing too much of an explanation, here's a beginning look at creating basic NumPy arrays.
\\\\
Covered in this section:
\begin{itemize}
	\item np.array: turn a list, or list of lists, into a NumPy array object
	\item shape: return the dimensions of an array object
	\item size: return the total number of elements in an array object
	\item save \& load: arrays can be saved for future use
\end{itemize}

\begin{python}
	x = np.array([1 ,2, 3, 4, 5])
	print(x)
	# [1 2 3 4 5]
	
	x.shape
	# (5,)
	
	Y = np.array([[1, 2, 3], [4, 5, 6], [7, 8, 9], [10, 11, 12]])
	print(Y)
	"""
	[[ 1  2  3]
	[ 4  5  6]
	[ 7  8  9]
	[10 11 12]]
	"""
	
	Y.shape
	# (4, 3)
	Y.size
	# 12
	
	# Notes Section:
	# dtype changes with mixed data, and will uptype (int + floats -> all floats) 
	# numpy arrays can be saved for later use
	
	# save
	np.save('my_array', x)
	# load
	y = np.load('my_array.npy')
\end{python}


\section{Exploration of Data}\label{sec:concept2}
We have a dataset on a topic or concept that has been deemed worthy for inspection! Surely, there are some insights to be gained from it. We load up the data, and then we hit a wall... Which columns are important? What questions can be answered? We can find the general statistics of the set, so what?
\\\\

This section will help with the initial process of data exploration, helping to find what is actually useful and should be examined further in the data. We'll start with single variables from the data and move into visually pairing multiple data points at once.
\\\\

We'll be using a few different Python libraries in this section:

\begin{python}
	import numpy as np
	import pandas as pd
	import matplotlib.pyplot as plt
	import seaborn as sb
\end{python}

\subsection{Univariate Exploration of Data}\label{sec:concept2.1}
% Concept2.1: Univariate Exploration of Data

\newpage
\subsection{Bivariate Exploration of Data}\label{sec:concept2.2}
% Concept2.2: Bivariate Exploration of Data
Having seen some basic ways to visually examine univariate data, we can move into exploring a data set by comparing two variables at a time with bivariate exploration of data.
\\\\
Commands:
\begin{itemize}
	\item plt.scatter: Matplotlib's command for a scatter plot.
	\item sb.regplot: Seaborn's command for a scatter plot. Much like sb's histogram default, their plot includes a line of best fit.
	\item plt.hist2d: Matplotlib's command for a heat map.
	\item sb.violinplot: Seaborn's command for a violin plot.
	\item sb.boxplot: Seaborn's command for a box plot.
	\item Creating Facets:
	\begin{itemize}
		\item g = sb.FacetGrid: Seaborn's initial command for creating facets (creates object to be called in secondary command). Specify category to create facets around.
		\item g.map: Seaborn's secondary command for creating facets. Specify type of plot and quantitative (usually) data to create facets around.
	\end{itemize}
	\item sb.barplot: Seaborn's bar plot command (not to be confused with sb.countplot). For use when pairing categorical values with quantitative values.
	\item sb.pointplot: Seaborn's command for creating a line plot.
\end{itemize}

Terminology:
\begin{itemize}
	\item heat maps: Use for discrete vs. discrete variables. Conceptualize as a top down 2-dimensional histogram. Can be an alternative to solving overplotting with a transparency method.
	\item scatter plots: Use for quantitative vs. quantitative variables.
	\item violin plots: Use for quantitative vs. qualitative variables. Plots a line from the minimum to maximum of the quantitative values, and features symmetrical smooth curves on both sides of the line which are wider at denser locations.
	\item box plots: Use for quantitative vs. qualitative variables. Plots a line from the minimum to maximum of the quantitative values, with a box drawn around the interquartile range [Q1:25th percentile, Q3; 75th percentile].
	\item clustered bar charts: Use for qualitative vs. quantitative variables. Compares variables by plotting aspects of categories side by side.
	\item faceting: creating a series of plots which share the same set of axes, but each subplot shows a subset of the data.
	\item line plots: A scatter plot where point estimates are marked for each of the categories with a given confidence interval. Also known as a time series plot.
	\item overplotting: When data or labels in a data visualization overlap, making it difficult to see individual points.
	\item sampling: Choose a subset of the data (best if randomized).
	\item transparency: By making the points partially transparent (increase the opacity), overplotting becomes evident by darker regions.
	\item jitter: By adding small randomly generated numbers to the values prior to plotting, overplotting becomes event by denser regions.
\end{itemize}

Although univariate data seems simple by virtue of a single variable, due to practicality, most people are likely more familiar with the first example shown in this section, a scatter plot. This is the classic x vs. y, plot your data using coordinates plot. A very effective tool, and easy to implement when the data consists of a single dependent variable. However, it's not the only tool! Datasets of different complexity and data types require different approaches to understand their stories. Let's dive in with an example.

\begin{python}
	# first things first, let's check out our dataset
	fuel_econ = pd.read_csv('fuel-econ.csv')
	fuel_econ.shape
	# (3929, 20)
	fuel_econ.head(10)
\end{python}

\begin{center}
\textbf{fuel\_econ.head(10)}
\end{center}
%\begin{center}
	\begin{tabular}{lrllrl}
		{} &     id &        make &           model &  year &           VClass \\
		0 &  32204 &      Nissan &            GT-R &  2013 &  Subcompact Cars \\
		1 &  32205 &  Volkswagen &              CC &  2013 &     Compact Cars \\
		2 &  32206 &  Volkswagen &              CC &  2013 &     Compact Cars \\
		3 &  32207 &  Volkswagen &      CC 4motion &  2013 &     Compact Cars \\
		4 &  32208 &   Chevrolet &  Malibu eAssist &  2013 &     Midsize Cars \\
		5 &  32209 &       Lexus &          GS 350 &  2013 &     Midsize Cars \\
		6 &  32210 &       Lexus &      GS 350 AWD &  2013 &     Midsize Cars \\
		7 &  32214 &     Hyundai &   Genesis Coupe &  2013 &  Subcompact Cars \\
		8 &  32215 &     Hyundai &   Genesis Coupe &  2013 &  Subcompact Cars \\
		9 &  32216 &     Hyundai &   Genesis Coupe &  2013 &  Subcompact Cars \\
	\end{tabular}
	
%\end{center}
\begin{tabular}{llllrr}
	{} &              drive &              trans &          fuelType &  cylinders &  displ \\
	0 &    All-Wheel Drive &    Automatic (AM6) &  Premium Gasoline &          6 &    3.8 \\
	1 &  Front-Wheel Drive &  Automatic (AM-S6) &  Premium Gasoline &          4 &    2.0 \\
	2 &  Front-Wheel Drive &     Automatic (S6) &  Premium Gasoline &          6 &    3.6 \\
	3 &    All-Wheel Drive &     Automatic (S6) &  Premium Gasoline &          6 &    3.6 \\
	4 &  Front-Wheel Drive &     Automatic (S6) &  Regular Gasoline &          4 &    2.4 \\
	5 &   Rear-Wheel Drive &     Automatic (S6) &  Premium Gasoline &          6 &    3.5 \\
	6 &    All-Wheel Drive &     Automatic (S6) &  Premium Gasoline &          6 &    3.5 \\
	7 &   Rear-Wheel Drive &    Automatic 8-spd &  Premium Gasoline &          4 &    2.0 \\
	8 &   Rear-Wheel Drive &       Manual 6-spd &  Premium Gasoline &          4 &    2.0 \\
	9 &   Rear-Wheel Drive &    Automatic 8-spd &  Premium Gasoline &          6 &    3.8 \\
\end{tabular}

\begin{tabular}{lrrrrr}
	{} &  pv2 &  pv4 &     city &    UCity &  highway \\
	0 &   79 &    0 &  16.4596 &  20.2988 &  22.5568 \\
	1 &   94 &    0 &  21.8706 &  26.9770 &  31.0367 \\
	2 &   94 &    0 &  17.4935 &  21.2000 &  26.5716 \\
	3 &   94 &    0 &  16.9415 &  20.5000 &  25.2190 \\
	4 &    0 &   95 &  24.7726 &  31.9796 &  35.5340 \\
	5 &    0 &   99 &  19.4325 &  24.1499 &  28.2234 \\
	6 &    0 &   99 &  18.5752 &  23.5261 &  26.3573 \\
	7 &   89 &    0 &  17.4460 &  21.7946 &  26.6295 \\
	8 &   89 &    0 &  20.6741 &  26.2000 &  29.2741 \\
	9 &   89 &    0 &  16.4675 &  20.4839 &  24.5605 \\
\end{tabular}

\begin{tabular}{lrrrrr}
	{} &  UHighway &     comb &  co2 &  feScore &  ghgScore \\
	0 &   30.1798 &  18.7389 &  471 &        4 &         4 \\
	1 &   42.4936 &  25.2227 &  349 &        6 &         6 \\
	2 &   35.1000 &  20.6716 &  429 &        5 &         5 \\
	3 &   33.5000 &  19.8774 &  446 &        5 &         5 \\
	4 &   51.8816 &  28.6813 &  310 &        8 &         8 \\
	5 &   38.5000 &  22.6002 &  393 &        6 &         6 \\
	6 &   36.2109 &  21.4213 &  412 &        5 &         5 \\
	7 &   37.6731 &  20.6507 &  432 &        5 &         5 \\
	8 &   41.8000 &  23.8235 &  375 &        6 &         6 \\
	9 &   34.4972 &  19.3344 &  461 &        4 &         4 \\
\end{tabular}
\newpage

Next, we'll start off with our first bivariate plot, a scatterplot comparing size of the engine, 'disp (L)', to the combined city and highway fuel efficiency, 'comb (mpg)'.

\begin{python}
	plt.scatter(data = fuel_econ, x = 'displ', y = 'comb')
	plt.xlabel('Displacement (l)')
	plt.ylabel('Combined Fuel Eff. (mpg)')
	# See Figure 15
	
	# seaborn's version incudes a line
	sb.regplot(data = fuel_econ, x = 'displ', y = 'comb', fit_reg = True)
	plt.xlabel('Displacement (l)')
	plt.ylabel('Combined Fuel Eff. (mpg)')
	# See Figure 16
\end{python}

\begin{figure}
	\includegraphics{images/figure15.png}
	\caption{Basic Matplotlib Scatter Plot.}\label{fig:figure15}
\end{figure}

\begin{figure}
	\includegraphics{images/figure16.png}
	\caption{Basic Seaborn Scatter Plot.}\label{fig:figure16}
\end{figure}

The dataset we're using has almost 4,000 entries. It will likely be the case that future datasets will have many more elements. Even though we're treating the displacement variable (volume in the engine in Liters) as a quantitative variable, it could almost be expressed as a categorical variable, although it is technically a discrete quantitative variable.

\begin{python}
	fuel_econ['displ'].value_counts().size
	# 46 different Liter options in this dataset
	fuel_econ['displ'].value_counts().head(10)
	"""
	2.0    907
	3.0    515
	3.5    245
	1.6    243
	3.6    208
	1.8    189
	2.5    169
	2.4    142
	1.4    136
	4.4    130
	"""
\end{python}

Although there will be different combinations when paired with the fuel efficiency variable, we've still stacked over 900 entries on x = 2, with much of the data points overlapping each other. This is a prime example of overplotting. How do we fix this?
\\\\

There are a few main methods to fix overplotting:

\begin{itemize}
	\item sampling: Choose a subset of the data (best if randomized).
	\item transparency: By making the points partially transparent (increase the opacity), overplotting becomes evident by the darker regions.
	\item jitter: By adding small randomly generated numbers to the values prior to plotting, overplotting becomes event by denser regions. 
	\item heat map: think of this as a 2-dimensional histogram where denser regions take on different colors.
\end{itemize}

In this example, we'll use Seaborn's scatter plot capabilities to add jitter and transparency, where transparency level is denoted with alpha. In a slight change of plot, we're mapping year onto overall fuel efficiency. Year is clearly another true example of discrete data.

\begin{python}
	sb.regplot(data = fuel_econ, x = 'year', y = 'comb', x_jitter = 0.3, scatter_kws = {'alpha' : 1/20})
	# See Figure 17
\end{python}

\begin{figure}
	\includegraphics{images/figure17.png}
	\caption{Seaborn Scatter Plot with jitter and transparency.}\label{fig:figure17}
\end{figure}

\newpage
Thought we were done with histograms? Let's check them in 2-dimensions with heat maps. Instead of changing the settings on a scatter plot, we could use another type of plot called a heat map. Similar to histograms, we need to be thoughtful with bin selection. So, start with the descriptive statistics.

\textbf{Statistics for Heat Map Guidance}
\begin{center}
	\begin{tabular}{lrr}
		{} &        displ &         comb \\
		count &  3929.000000 &  3929.000000 \\
		mean  &     2.950573 &    24.791339 \\
		std   &     1.305901 &     6.003246 \\
		min   &     0.600000 &    12.821700 \\
		25\%   &     2.000000 &    20.658100 \\
		50\%   &     2.500000 &    24.000000 \\
		75\%   &     3.600000 &    28.227100 \\
		max   &     7.000000 &    57.782400 \\
	\end{tabular}
\end{center}

\begin{python}
	bins_x = np.arange(0.6, 7 + 0.3, 0.3)
	bins_y = np.arange(12, 58 + 3, 3)
	plt.hist2d(data = fuel_econ, x = 'displ', y = 'comb', cmin = 0.5, cmap = 'viridis_r', bins = [bins_x, bins_y])
	plt.colorbar()
	plt.xlabel('Displacement (l)')
	plt.ylabel('Combined Fuel Eff. (mpg)')
	# See Figure 18
\end{python}

\begin{figure}
	\includegraphics{images/figure18.png}
	\caption{Basic Heat Map.}\label{fig:figure18}
\end{figure}

Not all inclusive, but to cover a major base, let's use a scatter plot for a continuous quantitative variable vs. a continuous quantitative variable.

\begin{python}
	sb.regplot(data = fuel_econ, x = 'city', y = 'highway', 
	scatter_kws = {'alpha' : 1/20})
	# See Figure 19
\end{python}

\begin{figure}
	\includegraphics{images/figure19.png}
	\caption{City vs. Highway Scatter Plot.}\label{fig:figure19}
\end{figure}

Next up is pairing categorical data with quantitative data. There are two useful tools, which seem very similar, but can help tell different stories.

\begin{itemize}
	\item violin plot
	\item box plot
\end{itemize}

One of the main differences between the two is that violin plots are wider where the data is more concentrated, which can be useful to see if data has more than one area of concentration (multimodal). Box plots shouldn't be down played. They are strong in placing emphasis on the interquartile range, and are known to be more simplistic and focused. Let's take a look!

\begin{python}
	# Violin Plots
	sb.violinplot(data = fuel_econ, x = 'VClass', y = 'comb', color = base_color, inner = None)
	plt.xticks(rotation = 15)
	# See Figure 20
	
	# Box Plots (simplicity + focus, as compared to the violin plot)
	sb.boxplot(data = fuel_econ, x = 'VClass', y = 'comb', color = base_color)
	plt.xticks(rotation = 15)
	# See Figure 21
\end{python}

\begin{figure}
	\includegraphics{images/figure20.png}
	\caption{Basic Violin Plot.}\label{fig:figure20}
\end{figure}

\begin{figure}
	\includegraphics{images/figure21.png}
	\caption{Basic Box Plot.}\label{fig:figure21}
\end{figure}

\newpage
A quick aside:
\\\\

It should absolutely be noted that with the prior, current and future plots illustrated, there are a plethora of options with each command. How should we handle the outliers? Can we transpose the IQR onto a violin plot? How can we focus on these specific data points? It's highy recommended to at least scan through at least some of the library and function documentation.
\\\\

Probably the simplest, yet valuable, plot yet is a clustered bar chart. This takes our bar chart plot from the univariate exploration section, and simply adds more bars. These can be either different categories, or can be aspects of the same category.
\\\\

For a quick example of a clustered bar chart, we'll continue on with our examination of Vehicle Class ('VClass'). As shown in the violin and box plots, there are 5 different types of vehicles in the dataset. However, we can break down the types of vehicles into subsets based on their transmission types (Automatic vs. Manual).

\begin{python}
	# a bit of data cleaning: just need auto vs. manual from the strings
	fuel_econ['trans_type'] = fuel_econ['trans'].apply(lambda x: x.split()[0])
	sb.countplot(data = fuel_econ, x = 'VClass', hue = 'trans_type')
	plt.xticks(rotation = 15)
	# See Figure 22
\end{python}

\begin{figure}
	\includegraphics{images/figure22.png}
	\caption{Vertical Clustered Bar Chart.}\label{fig:figure22}
\end{figure}

Let's do another example of a clustered bar chart that puts together different aspects. In this example, we want to investigate the difference between the two main fuel types (regular and premium) by vehicle class.

\begin{python}
	fuel_econ_subset = fuel_econ.loc[fuel_econ['fuelType'].isin(['Premium Gasoline', 'Regular Gasoline'])]
	sb.countplot(data = fuel_econ_subset, y = 'VClass', hue = 'fuelType')
	# See Figure 23
\end{python}

\begin{figure}
	\includegraphics{images/figure23.png}
	\caption{Vertical Clustered Bar Chart.}\label{fig:figure23}
\end{figure}

\newpage
With even more plotting techniques in our visualization toolbox, what happens when it seems sensible to test many different variations of a similar plot against the data, can we avoid creating a single plot at a time? Enter faceting, creating side by side series of plots containing subsets of the data.
\\\\

For an example, let's create a plot for every type of vehicle paired with combined efficiency.

\begin{python}
	bins = np.arange(12, 58 + 2, 2)
	g = sb.FacetGrid(data = fuel_econ, col = 'VClass', col_wrap = 3, sharey = False)
	g.map(plt.hist, 'comb', bins = bins)
	# See Figure 24
\end{python}

\begin{figure}
	\includegraphics[width=\textwidth,height=\textheight,keepaspectratio]{images/figure24.png}
	\caption{Vehicle Class Fuel Efficiency Facets.}\label{fig:figure24}
\end{figure}

\newpage
One last plot to show for this section is a line plot, which shows a point estimate along with a confidence interval while comparing categorical variables with quantitative data.

\begin{python}
	sb.pointplot(data = fuel_econ, x = 'VClass', y = 'comb')
	plt.xticks(rotation = 15)
	plt.ylabel('Avg. Combined Fuel Eff. (mpg)')
	# See Figure 25
\end{python}

\begin{figure}
	\includegraphics{images/figure25.png}
	\caption{Line Plot.}\label{fig:figure25}
\end{figure}

\newpage
\subsection{Multivariate Exploration of Data}\label{sec:concept2.3}
% Concept2.3: Multivariate Exploration of Data

\newpage
\section{Explanatory Visualizations}\label{sec:concept3}
% Concept3: Explanatory Visuals

\section{Visualization Case Study}\label{sec:concept4}
% Concept4: Visualization Case Study
	
\end{document}