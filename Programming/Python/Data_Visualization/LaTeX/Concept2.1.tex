% Concept2.1: Univariate Exploration of Data
The first tools added into our data exploration toolkit will be for univariate data, or examining the data a single variable at a time. Even though we're going to be using several different libraries, we'll still try to keep track of new and essential commands and terminology:
\\\\

Commands:
\begin{itemize}
	\item df.head: Retrieves the first few rows of data from a Pandas DataFrame. Able to specify how many rows
	\item sb.countplot: Seaborn's command for generating a bar chart.
	\item df[column].value\_counts().index: Will return an immutable sequence sorted number of categorical entries (highest to lowest).
	\item df.melt: Will unpivot a DataFrame from wide to long format (i.e. we can "melt" two entries together).
	\item plt.hist: Matplotlib's command for generating a histogram.
	\item sb.distplot: Seaborn's command for generating a histogram. Default command also includes a density curve estimation.
\end{itemize}

Terminology:
\begin{itemize}
	\item bar chart: Useful to show counts across categories.
	\item relative frequency: Shows proportion of each category in population.
	\item pie chart: Use these to show how whole of data is broken down into parts, useful when plotting a small number of slices (usually no more than four).
	\item histogram: Useful to examine quantitative data.
\end{itemize}

We'll be using a Pokemon dataset for our examples. Let's get an idea of what our data looks like:

\begin{python}
	pokemon = pd.read_csv("pokemon.csv")
	pokemon.shape
	# pokemon.shape = (807, 14)
	pokemon.head(10)
\end{python}

\newpage
\textbf{Pokemon.Head(10)}
\begin{center}
	\begin{tabular}{lrlrrrrl}
		{} &  id &     species &  generation\_id &  height &  weight &  base\_experience & type\_1 \\
		0 &   1 &   bulbasaur &              1 &     0.7 &     6.9 &               64 &  grass \\
		1 &   2 &     ivysaur &              1 &     1.0 &    13.0 &              142 &  grass \\
		2 &   3 &    venusaur &              1 &     2.0 &   100.0 &              236 &  grass \\
		3 &   4 &  charmander &              1 &     0.6 &     8.5 &               62 &   fire \\
		4 &   5 &  charmeleon &              1 &     1.1 &    19.0 &              142 &   fire \\
		5 &   6 &   charizard &              1 &     1.7 &    90.5 &              240 &   fire \\
		6 &   7 &    squirtle &              1 &     0.5 &     9.0 &               63 &  water \\
		7 &   8 &   wartortle &              1 &     1.0 &    22.5 &              142 &  water \\
		8 &   9 &   blastoise &              1 &     1.6 &    85.5 &              239 &  water \\
		9 &  10 &    caterpie &              1 &     0.3 &     2.9 &               39 &    bug \\
	\end{tabular}
\end{center}
\begin{center}
	\begin{tabular}{llrrrrrr}
		{} &  type\_2 &  hp &  attack &  defense &  speed &  special-attack &  special-defense \\
		0 &  poison &  45 &      49 &       49 &     45 &              65 &               65 \\
		1 &  poison &  60 &      62 &       63 &     60 &              80 &               80 \\
		2 &  poison &  80 &      82 &       83 &     80 &             100 &              100 \\
		3 &     NaN &  39 &      52 &       43 &     65 &              60 &               50 \\
		4 &     NaN &  58 &      64 &       58 &     80 &              80 &               65 \\
		5 &  flying &  78 &      84 &       78 &    100 &             109 &               85 \\
		6 &     NaN &  44 &      48 &       65 &     43 &              50 &               64 \\
		7 &     NaN &  59 &      63 &       80 &     58 &              65 &               80 \\
		8 &     NaN &  79 &      83 &      100 &     78 &              85 &              105 \\
		9 &     NaN &  45 &      30 &       35 &     45 &              20 &               20 \\
	\end{tabular}
\end{center}

Now that we've had a peak at the data, let's get into some visualization.

\begin{python}
	# create a bar chart for one of the columns (generation_id)
	# this will show counts of each pokemon for each generation
	sb.countplot(data = pokemon, x = 'generation_id')
	# See Figure 1
	
	# let's use a single neutral color to reduce redundancy and make it easier on the eyes
	base_color = sb.color_palette()[0]
	sb.countplot(data = pokemon, x = 'generation_id', color = base_color)
	# See Figure 2
\end{python}

%\newpage
\begin{figure}
	\includegraphics{images/figure1.png}
	\caption{Generic bar chart.}\label{fig:figure1}
\end{figure}

\begin{figure}
	\includegraphics{images/figure2.png}
	\caption{Bar chart with neutral coloring.}\label{fig:figure2}
\end{figure}

A simple, yet effective visual trick is adding some order to your plots via sorting.

\begin{python}
	# let's sort the data highest -> lowest
	# we can use pandas series function: value_counts()
	gen_order = pokemon['generation_id'].value_counts().index
	sb.countplot(data = pokemon, x = 'generation_id', color = base_color, order = gen_order)
	# See Figure 3
\end{python}

\begin{figure}
	\includegraphics{images/figure3.png}
	\caption{Bar chart with neutral coloring and ordering applied.}\label{fig:figure3}
\end{figure}

Let's combine the previous coloring and sorting to look at a new variable. Additionally, since we know the x-axis labels are somewhat long. Two favorable ways of dealing with this is to either rotate the x-axis labels, or create a horizontal bar chart.

\begin{python}
	type_order = pokemon['type_1'].value_counts().index
	sb.countplot(data = pokemon, x = 'type_1', color = base_color, order = type_order)
	plt.xticks(rotation = 90) 
	# See Figure 4
	
	# Change x to y to change from vertical to horizontal bar chart
	sb.countplot(data = pokemon, y = 'type_1', color = base_color, order = type_order)
	# See Figure 5
\end{python}

\begin{figure}
	\includegraphics{images/figure4.png}
	\caption{Standard bar chart format with x-axis labels rotated.}\label{fig:figure4}
\end{figure}

\begin{figure}
	\includegraphics{images/figure5.png}
	\caption{Standard bar chart format in vertical form.}\label{fig:figure5}
\end{figure}

The previous examples display actual counts in the bar charts, but it can also be useful to display data using relative frequency (proportion of each category in the population). This gives us an opportunity to introduce a few other concepts as well. Most importantly, we'll introduce the melt function, which we'll use to combine two categories (two distinct columns) into a single category.

\begin{python}
	pkmn_types = pokemon.melt(id_vars=['id', 'species'], 
				 value_vars=['type_1', 'type_2'], 
				 var_name='type_level', 
				 value_name='type')
	"""
	Let's walk through how the melt function is being used here:
	- id_vars: columns to utilize as identifier variables
	- value_vars: columns to unpivot
	- var_name: name of column to describe data being unpivoted together
	- value_name: name of column to match values from the unpivoted labels
	
	Note: the pokemon DataFrame has 807 rows (pokemon entries). This use of melt puts two categories together into a single category. Thus, each pokemon now has two entries each, making the pkmn_types DataFrame have double the rows as the original DataFrame or 1614 rows.
	"""
	
	base_color = sb.color_palette()[0]
	type_counts = pkmn_types['type'].value_counts()
	type_order = type_counts.index
	sb.countplot(data = pkmn_types, y = 'type', color = base_color, order = type_order)
	# See Figure 6
	
	# now we'll change the tick counts
	n_pokemon = pokemon.shape[0]
	max_type_count = type_counts[0]
	max_prop = max_type_count / n_pokemon
	# note: max_prop = 0.16
	
	tick_props = np.arange(0, max_prop, 0.02)
	tick_names = ['{:0.2f}'.format(v) for v in tick_props]
	"""
	tick_props creates the numbers for an equally spaced axis, while tick_names takes the numeric version and returns a list of a string values.
	"""
	
	# The culmination of the above work into a chart:
	sb.countplot(data = pkmn_types, y = 'type', color = base_color, order = type_order)
	plt.xticks(tick_props * n_pokemon, tick_names)
	plt.xlabel('proportion')
	for j in range(type_counts.shape[0]):
	count = type_counts[j]
	pct_string = '{:0.1f}%'.format(100*count/n_pokemon)
	plt.text(count + j, j, pct_string, va = 'center')
	# See Figure 7
\end{python}

\begin{figure}
	\includegraphics{images/figure6.png}
	\caption{Bar chart of melted pokemon types.}\label{fig:figure6}
\end{figure}

\begin{figure}
	\includegraphics{images/figure7.png}
	\caption{Bar chart of melted pokemon types shown in relative frequency.}\label{fig:figure7}
\end{figure}

Whereas bar charts are a great initial step in exploring categorical data, histogram are useful in the exploration of quantitative data. Let's take a look at our variables in the pokemon DataFrame which contain quantitative data.

\begin{python}
	# basic histogram exploring the speed variable
	plt.hist(data = pokemon, x = 'speed', bins = 20)
	# See Figure 8
	
	# for historgrams, bin specification is paramount in visualization
	# this is an example
	bins = np.arange(0, pokemon['speed'].max() + 5, 5)
	# recall that arange will not include last value
	plt.hist(data = pokemon, x = 'speed', bins = bins)
	# See Figure 9
	
	# Seaborn's histogram command has a default density curve estimation
	sb.distplot(pokemon['speed'])
	# See Figure 10
	
\end{python}

\begin{figure}
	\includegraphics{images/figure8.png}
	\caption{Basic histogram of Pokemon speeds.}\label{fig:figure8}
\end{figure}

\begin{figure}
	\includegraphics{images/figure9.png}
	\caption{Histogram of Pokemon speeds with altered bins.}\label{fig:figure9}
\end{figure}

\begin{figure}
	\includegraphics{images/figure10.png}
	\caption{Seaborn's default histogram.}\label{fig:figure10}
\end{figure}

Earlier in the course, we mentioned avoiding a high "lie factor". One case in which it may actually be beneficial to "zoom" in on data is when dealing with outliers, or when a large majority of the data is within certain axis limits.
\\\\

For example, let's take a look at the height variable of Pokemon.

\begin{python}
	# even though this is a course on visuals, basic math and descriptive statistics can still be beneficial
	pokemon['height'].describe()
	"""
	count    807.000000
	mean       1.162454
	std        1.081030
	min        0.100000
	25%        0.600000
	50%        1.000000
	75%        1.500000
	max       14.500000
	
	With the majority of our data between 0 and 1.5, it's acceptible to change our x-axis limits. The outlier of 14.5 would make our histogram and bin choice less effective from a visual sense.
	"""
	bins = np.arange(0, pokemon['height'].max() + 0.2, 0.2)
	plt.hist(data = pokemon, x = 'height', bins = bins)
	plt.xlim((0, 6))
	# See Figure 11
\end{python}

\begin{figure}
	\includegraphics{images/figure11.png}
	\caption{Histogram with altered x-axis limits.}\label{fig:figure11}
\end{figure}

Another method when dealing with outliers or data points that have very large differences between them is with scaling and transformations.
\\\\
Let's take a look at the weight variable of Pokemon.

\begin{python}
	pokemon['weight'].describe()
	"""
	count    807.000000
	mean      61.771128
	std      111.519355
	min        0.100000
	25%        9.000000
	50%       27.000000
	75%       63.000000
	max      999.900000
	
	Obviously, there exists some outlier(s) here. We should think about using transformations and tick-mark manipulation to visualize the data this time around.
	"""
	
	# Let's see what we're dealing with without any scaling or transformations
	bins = np.arange(0, pokemon['weight'].max() + 40, 40)
	plt.hist(data = pokemon, x = 'weight', bins = bins)
	# See Figure 12
	"""
	Even with specifying the bins, the histrogram proves to not be too insightful. Changing the bins alone likey won't solve this issue.
	"""
	
	# Let's try applying a transformation of the x-scale, itself
	bins = np.arange(0, pokemon['weight'].max() + 40, 40)
	plt.hist(data = pokemon, x = 'weight', bins = bins)
	plt.xscale('log')
	# See Figure 13
	"""
	It could be argued this is a worse approach, but using a logarithmic transformation could steer us in the right direction.
	"""
	
	np.log10(pokemon['weight'].describe())
	"""
	count    2.906874
	mean     1.790786
	std      2.047350
	min     -1.000000
	25%      0.954243
	50%      1.431364
	75%      1.799341
	max      2.999957
	
	Applying a logarithmic transformation on the data, itself, appears to shrinks the numbers into a more manageable scale.
	"""
	
	# Get the ticks for bins between [0 - maximum weight]
	bins = 10 ** np.arange(-1, 3 + 0.1, 0.1)
	
	# Generate the x-ticks we want to apply
	ticks = [0.1, 0.3, 1, 3, 10, 30, 100, 300, 1000]
	"""
	Important: note here how we are using differently spaced tick marks in the original scale.
	"""
	
	# Convert ticks into string values, to be displayed along the x-axis
	labels = ['{}'.format(v) for v in ticks]
	
	# Plot the histogram
	plt.hist(data=pokemon, x='weight', bins=bins);
	
	# The argument in the xscale() represents the axis scale type to apply.
	# The possible values are: {"linear", "log", "symlog", "logit", ...}
	plt.xscale('log')
	
	# Apply x-ticks
	plt.xticks(ticks, labels)
	
	# See Figure 14
\end{python}

\begin{figure}
	\includegraphics{images/figure12.png}
	\caption{First try histogram of Pokemon's weight.}\label{fig:figure12}
\end{figure}

\begin{figure}
	\includegraphics{images/figure13.png}
	\caption{Histogram of Pokemon's weight with a log transformation applied to the x-scale.}\label{fig:figure13}
\end{figure}

\begin{figure}
	\includegraphics{images/figure14.png}
	\caption{Histogram of Pokemon's weight with transformation and scaling applied.}\label{fig:figure14}
\end{figure}