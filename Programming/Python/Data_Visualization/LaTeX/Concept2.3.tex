% Concept2.3: Multivariate Exploration of Data
We'll be making use of both the pokemon and fuel\_econ datasets while discussing exploration of multivariate data. When plotting three variables together, there are four major cases to consider:
\begin{itemize}
	\item three numeric variables
	\item two numeric variables and one categorical variable
	\item one numberic variable and two categorical variables
	\item three categorical variables
\end{itemize}

Commands:
\begin{itemize}
	\item np.random.seed: Use when you want "randomized" results to be reproducible.
	\item np.random.choice: If an ndarray is passed through, a random sampling is produced from its data. If an integer is passed through, a random sampling is generated as if it were np.arange.
	\item sb.PairGrid: Seaborn function for Plot Matrices (i.e. plot all pairwise relationships among data for specified variables).
	\item sb.PairGrid.map\_offdiag: Will plot onto currently active matplotlib axes (i.e. will draw over the current diagonal of the plot matrices).
\end{itemize}

Terminology:
\begin{itemize}
	\item Non-Positional Encoding (for Third Variables): Recall the main encodings of size, shape and color. If we have a plot which illustrates a relationship between two of the variables, we can use an encoding to illustrate another relationship.
	\item Plot Matrices: Plot pairwise relationships among data given specified variables.
	\item Feature Engineering: Creating a new variable by leveraging relationships between variables with mathematical operations.
\end{itemize}

We'll start with an example illustrating the effectiveness of non-positional encodings for a third variable.
\\\\

An example of using color as the encoder:
\begin{python}
	"""
	We'll start by taking a random sampling from the fuel_econ dataset. Note that since this a NumPy function, an ndarray must be passed through. A Pandas DataFrame will not work, which is why we must use two steps.
	"""
	np.random.seed(2018)
	sample = np.random.choice(fuel_econ.shape[0], 200, replace = False)
	fuel_econ_subset = fuel_econ.loc[sample]
	
	ttype_markers = [['Automatic', 'o'], ['Manual', '^']]
	
	for ttype, marker in ttype_markers:
		plot_data = fuel_econ_subset.loc[fuel_econ_subset['trans_type'] == ttype]
		sb.regplot(data = fuel_econ_subset, x = 'displ', y = 'comb', x_jitter = 0.04, fit_reg = False, marker = marker)
	plt.xlabel('Displacement (l)')
	plt.ylabel('Combined Fuel Eff. (mpg)')
	# See Figure 29
\end{python}

\begin{figure}
	\includegraphics{images/figure29.png}
	\caption{Non-Positional Encoding Example: Shape.}\label{fig:figure29}
\end{figure}

An example of using size as an encoder:

\begin{python}
	sizes = [200, 350, 500]
	legend_obj = []
	for s in sizes:
		legend_obj.append(plt.scatter([],[],s = s/2, color = base_color))
	plt.legend(legend_obj, sizes, title = 'CO2 (g/mi)')
	sb.regplot(data = fuel_econ_subset, x = 'displ', y = 'comb', x_jitter = 0.04, fit_reg = False, scatter_kws = {'s' : fuel_econ_subset['co2']/2})
	plt.xlabel('Displacement (l)')
	plt.ylabel('Combined Fuel Eff. (mpg)')
	# See Figure 30
\end{python}

\begin{figure}
	\includegraphics{images/figure30.png}
	\caption{Non-Positional Encoding Example: Size.}\label{fig:figure30}
\end{figure}

\newpage
As we could've guessed, when it comes to encodings, color is more natural. Here are three different examples which use color encoding to represent a third variable.

\begin{python}
	"""
	Note: FacetGrid is used differently in the first two examples. Rather than create a series of plots, its used to separate categorical data from the third variable.
	"""
	
	g = sb.FacetGrid(data = fuel_econ_subset, hue = 'trans_type', hue_order = ['Automatic', 'Manual'], size = 4, aspect = 1.5)
	g.map(sb.regplot, 'displ', 'comb', x_jitter = 0.04, fit_reg = False)
	g.add_legend()
	plt.xlabel('Displacement (l)')
	plt.ylabel('Combined Fuel Eff. (mpg)')
	# See Figure 31
	
	g = sb.FacetGrid(data = fuel_econ_subset, hue = 'VClass', size = 4, aspect = 1.5, palette = 'viridis_r')
	g.map(sb.regplot, 'displ', 'comb', x_jitter = 0.04, fit_reg = False)
	g.add_legend()
	plt.xlabel('Displacement (l)')
	plt.ylabel('Combined Fuel Eff. (mpg)')
	# See Figure 32
	
	plt.scatter(data = fuel_econ_subset, x = 'displ', y = 'comb', c = 'co2', cmap = 'viridis_r')
	plt.colorbar(label = 'CO2 (g/mi)')
	plt.xlabel('Displacement (l)')
	plt.ylabel('Combined Fuel Eff. (mpg)')
	# See Figure 33
\end{python}

\begin{figure}
	\includegraphics[width=\textwidth,height=\textheight,keepaspectratio]{images/figure31.png}
	\caption{Non-Positional Encoding Example: Color \#1.}\label{fig:figure31}
\end{figure}

\begin{figure}
	\includegraphics[width=\textwidth,height=\textheight,keepaspectratio]{images/figure32.png}
	\caption{Non-Positional Encoding Example: Color \#2.}\label{fig:figure32}
\end{figure}

\begin{figure}
	\includegraphics{images/figure33.png}
	\caption{Non-Positional Encoding Example: Color \#3.}\label{fig:figure33}
\end{figure}

Recall the pokemon dataset, and how we previously unpivoted the type\_1 and type\_2 data together? Let's resurrect that code and mix encodings with transformations to examine relationships between height, weight and type.

\begin{python}
	# create our single column of types
	type_cols = ['type_1','type_2']
	non_type_cols = pokemon.columns.difference(type_cols)
	pkmn_types = pokemon.melt(id_vars = non_type_cols, value_vars = type_cols, var_name = 'type_level', value_name = 'type').dropna()
	
	# apply a log transformation to the data we're examining
	pkmn_types['log_weight'] = np.log(pkmn_types['weight'])
	pkmn_types['log_height'] = np.log(pkmn_types['height'])
	
	# Let's specifically zoom in on fairy and dragon type pokemon
	ptypes = ['fairy', 'dragon']
	
	g = sb.FacetGrid(data = pkmn_types, hue = 'type', hue_order = ptypes)
	g.map(sb.regplot, 'log_weight', 'height', fit_reg = False)
	g.add_legend()
	plt.xlabel('log of Weight')
	plt.ylabel('Height')
	# See Figure 34
\end{python} 

\begin{figure}
	\includegraphics[width=\textwidth,height=\textheight,keepaspectratio]{images/figure34.png}
	\caption{Pokemon Height vs. Weight by Category.}\label{fig:figure34}
\end{figure}

With bivariate data, when we used faceting to create a series of plots, we used bar charts or histograms while changing a specification of the second variable. With three variables, we could use scatter plots comparing two variables with a third variable encoded. This is known as faceting in two directions.
\\\\

Let's go back to our fuel economy dataset and compare vehicle class, transmission type and combined fuel efficiency.

\begin{python}
	g = sb.FacetGrid(data = fuel_econ, col = 'VClass', row = 'trans_type')
	g.map(plt.scatter, 'displ', 'comb')
	# See Figure 35
\end{python}

\begin{figure}
	\includegraphics[width=\textwidth,height=\textheight,keepaspectratio]{images/figure35.png}
	\caption{Basic Faceting in Two Directions.}\label{fig:figure35}
\end{figure}

Continuing the trend, here are some more adaptations of bivariate plots extended to multivariate plots.

\begin{python}
	sb.pointplot(data = fuel_econ, x = 'VClass', y = 'comb', hue = 'trans_type', ci = 'sd', linestyles = '', dodge = True)
	plt.xticks(rotation = 15)
	plt.ylabel('Avg. Combined Efficiency (mpg)')
	# See Figure 36
	
	sb.barplot(data = fuel_econ, x = 'VClass', y = 'comb', hue = 'trans_type', ci = 'sd')
	plt.xticks(rotation = 15)
	plt.ylabel('Avg. Combined Efficiency (mpg)')
	# See Figure 37
	
	sb.boxplot(data = fuel_econ, x = 'VClass', y = 'comb', hue = 'trans_type')
	plt.xticks(rotation = 15)
	plt.ylabel('Avg. Combined Efficiency (mpg)')
	# See Figure 38
	
	# not illustrated, but feasible, is using a heatmap to illustrate a third variable
\end{python}

\begin{figure}
	\includegraphics{images/figure36.png}
	\caption{Pointplot for Multivariate.}\label{fig:figure36}
\end{figure}

\begin{figure}
	\includegraphics{images/figure37.png}
	\caption{Bar Plot for Multivariate.}\label{fig:figure37}
\end{figure}

\begin{figure}
	\includegraphics{images/figure38.png}
	\caption{Box Plot for Multivariate.}\label{fig:figure38}
\end{figure}

And a few more challenge examples:

\begin{python}
	# Plot the city vs highway fuel efficiencies for each vehicle class
	g = sb.FacetGrid(data = fuel_econ, col = 'VClass')
	g.map(plt.scatter, 'city', 'highway')
	# See Figure 39
	
	# Plot the relationship between engine size (displ), class, and fuel type (fuelType). Use premium vs. regular
	sb.barplot(data = fuel_econ, x = 'VClass', y = 'displ', hue = 'fuelType', hue_order = ['Premium Gasoline', 'Regular Gasoline'], ci = 'sd')
	plt.xticks(rotation = 15)
	plt.ylabel('Engine Size (l)')
	# See Figure 40
\end{python}

\begin{figure}
	\includegraphics[width=\textwidth,height=\textheight,keepaspectratio]{images/figure39.png}
	\caption{Vehicle Class vs. Combined Fuel Efficiency (by Transmission Type).}\label{fig:figure39}
\end{figure}

\begin{figure}
	\includegraphics{images/figure40.png}
	\caption{Vehicle Class vs. Engine Size (by Fuel Type).}\label{fig:figure40}
\end{figure}

Possibly even more powerful than a few different combinations would be pairwise combinations between specified variables. This is the concept of plot matrices.

\begin{python}
	# let's look at the pairwise relationships between the 6 combat types
	pkmn_stats = ['hp', 'attack', 'defense', 'speed', 'special-attack', 'special-defense']
	g = sb.PairGrid(data = pokemon, vars = pkmn_stats)
	g.map(plt.scatter)
	# See Figure 41
	
	# if needed, we even impose another plot over the existing plots on the diagonal
	g = g.map_offdiag(plt.scatter)
	g.map_diag(plt.hist)
	# See Figure 42
\end{python}

\begin{figure}
	\includegraphics[width=\textwidth,height=\textheight,keepaspectratio]{images/figure41.png}
	\caption{Pairwise Relationships.}\label{fig:figure41}
\end{figure}

\begin{figure}
	\includegraphics[width=\textwidth,height=\textheight,keepaspectratio]{images/figure42.png}
	\caption{Histograms on the Diagonal.}\label{fig:figure42}
\end{figure}

If we run a slight variaton of the code above, we'll actual just get the histogram along the diagonal, instead of imposed.

\begin{python}
	econ_vars = ['displ', 'co2', 'city', 'highway', 'comb']
	g = sb.PairGrid(data = fuel_econ, vars = econ_vars)
	
	"""
	If we don't run:
	g.map(plt.scatter)
	Then, just the histograms will be drawn on the diagonals.
	"""
	
	g = g.map_offdiag(plt.scatter)
	g.map_diag(plt.hist)
	# See Figure 43
\end{python}

\begin{figure}
	\includegraphics[width=\textwidth,height=\textheight,keepaspectratio]{images/figure43.png}
	\caption{Histograms on the Diagonal.}\label{fig:figure43}
\end{figure}

We'll draw the multivariate exploration to an end with one last concept, feature engineering. Let's work through a challenge problem to get an idea of how this can be used.
\\\\

The output of the preceding task pointed out a potentially interesting relationship between co2 emissions and overall fuel efficiency. Engineer a new variable that depicts CO2 emissions as a function of gallons of gas (g / gal). (The 'co2' variable is in units of g / mi, and the 'comb' variable is in units of mi / gal.) Then, plot this new emissions variable against engine size ('displ') and fuel type ('fuelType'). For this task, compare not just Premium Gasoline and Regular Gasoline, but also Diesel fuel.

\begin{python}
	fuel_econ['new_var'] = fuel_econ['co2'] * fuel_econ['comb']
	fuel_econ_sub = fuel_econ.loc[fuel_econ['fuelType'].isin(['Premium Gasoline', 'Regular Gasoline', 'Diesel'])]
	
	# plotting
	g = sb.FacetGrid(data = fuel_econ_sub, col = 'fuelType', size = 4, col_wrap = 3)
	g.map(sb.regplot, 'new_var', 'displ', y_jitter = 0.04, fit_reg = False, scatter_kws = {'alpha' : 1/5})
	g.set_ylabels('Engine displacement (l)')
	g.set_xlabels('CO2 (g/gal)')
	g.set_titles('{col_name}')
	# See Figure 44
\end{python}

\begin{figure}
	\includegraphics[width=\textwidth,height=\textheight,keepaspectratio]{images/figure44.png}
	\caption{Feature Engineering.}\label{fig:figure44}
\end{figure}