% Concept4: Visualization Case Study
We've built a solid toolkit for data visualization from the exploratory to the explanatory. We'll put all of this together in a final case study!
\\\\
We're given a dataset containing entries on about 54,000 diamonds with 10 variables on each diamond observation.

\begin{itemize}
	\item price
	\item carat
	\item cut
	\item color
	\item clarity
	\item x
	\item y
	\item z
	\item table
	\item depth
\end{itemize}

We haven't been given a specific task or goal with the provided dataset. So, let's just explore the variables and relationships.

\begin{python}
	diamonds = pd.read_csv("diamonds.csc")
	
	# high-level overview of data shape and composition
	diamonds.shape
	diamonds.dtypes
	diamonds.head(10)
	diamonds.describe()
\end{python}

The exact shape of the diamonds dataset was (53940, 10).

\textbf{diamonds.dtypes}
\begin{center}
	\begin{tabular}{ll}
		{} &        0 \\
		carat   &  float64 \\
		cut     &   object \\
		color   &   object \\
		clarity &   object \\
		depth   &  float64 \\
		table   &  float64 \\
		price   &    int64 \\
		x       &  float64 \\
		y       &  float64 \\
		z       &  float64 \\
	\end{tabular}
\end{center}

\textbf{diamonds.head(10)}
\begin{center}
	\begin{tabular}{ll}
		{} &        0 \\
		carat   &  float64 \\
		cut     &   object \\
		color   &   object \\
		clarity &   object \\
		depth   &  float64 \\
		table   &  float64 \\
		price   &    int64 \\
		x       &  float64 \\
		y       &  float64 \\
		z       &  float64 \\
	\end{tabular}
\end{center}

\textbf{diamonds.describe}
\begin{center}
	\footnotesize
	\begin{tabular}{lrrrrrrr}
		{} &         carat &         depth &         table &         price &             x &             y &             z \\
		count &  53940.000000 &  53940.000000 &  53940.000000 &  53940.000000 &  53940.000000 &  53940.000000 &  53940.000000 \\
		mean  &      0.797940 &     61.749405 &     57.457184 &   3932.799722 &      5.731157 &      5.734526 &      3.538734 \\
		std   &      0.474011 &      1.432621 &      2.234491 &   3989.439738 &      1.121761 &      1.142135 &      0.705699 \\
		min   &      0.200000 &     43.000000 &     43.000000 &    326.000000 &      0.000000 &      0.000000 &      0.000000 \\
		25\%   &      0.400000 &     61.000000 &     56.000000 &    950.000000 &      4.710000 &      4.720000 &      2.910000 \\
		50\%   &      0.700000 &     61.800000 &     57.000000 &   2401.000000 &      5.700000 &      5.710000 &      3.530000 \\
		75\%   &      1.040000 &     62.500000 &     59.000000 &   5324.250000 &      6.540000 &      6.540000 &      4.040000 \\
		max   &      5.010000 &     79.000000 &     95.000000 &  18823.000000 &     10.740000 &     58.900000 &     31.800000 \\
	\end{tabular}
\end{center}

\newpage
After the initial look through the dataset, we'll begin with some univariate exploration:
\begin{python}
	# Let's start by looking at price... is distribution skewed / symmetric? Unimodal or multimodal?
	base_color = sb.color_palette()[0]
	plt.hist(data = diamonds, x = 'price', bins = 150)
	plt.xscale('log')
	# See Figure 46
	
	# Look at carat
	plt.hist(data = diamonds, x = 'carat', bins = 50)
	plt.xlim((0, 4))
	# See Figure 47
	
	# Look at cut, color, and clarity grades (categorical variables)
	
	# cut
	cut_order = diamonds['cut'].value_counts().index
	sb.countplot(data = diamonds, x = 'cut', color = base_color, order = cut_order)
	# See Figure 48
	
	# color
	color_order = diamonds['color'].value_counts().index
	sb.countplot(data = diamonds, x = 'color', color = base_color, order = color_order)
	# See Figure 49
	
	# clarity
	clarity_order = diamonds['clarity'].value_counts().index
	sb.countplot(data = diamonds, x = 'clarity', color = base_color, order = clarity_order)
	# See Figure 50
	
	# Include general categorical variable comment
	# Include distribution comment about each category
\end{python}

%\begin{figure}
%	\includegraphics[width=\textwidth,height=\textheight,keepaspectratio]{images/figure44.png}
%	\caption{Feature Engineering.}\label{fig:figure44}
%\end{figure}

\begin{figure}
	\includegraphics{images/figure46.png}
	\caption{Price Histogram.}\label{fig:figure46}
\end{figure}

\begin{figure}
	\includegraphics{images/figure47.png}
	\caption{Carat Histogram.}\label{fig:figure47}
\end{figure}

\begin{figure}
	\includegraphics{images/figure48.png}
	\caption{Cut Bar Plot.}\label{fig:figure48}
\end{figure}

\begin{figure}
	\includegraphics{images/figure49.png}
	\caption{Color Bar Plot.}\label{fig:figure49}
\end{figure}

\begin{figure}
	\includegraphics{images/figure50.png}
	\caption{Clarity Bar Plot.}\label{fig:figure50}
\end{figure}

\newpage
Next, we'll move into bivariate exploration.

\begin{python}
	# Step 2: Bivariate Exploration
	# convert cut, color, and clarity into ordered categorical types
	ordinal_var_dict = {'cut': ['Fair','Good','Very Good','Premium','Ideal'],
		'color': ['J', 'I', 'H', 'G', 'F', 'E', 'D'],
		'clarity': ['I1', 'SI2', 'SI1', 'VS2', 'VS1', 'VVS2', 'VVS1', 'IF']}
	
	
	for var in ordinal_var_dict:
		pd_ver = pd.__version__.split(".")
		if (int(pd_ver[0]) > 0) or (int(pd_ver[1]) >= 21): # v0.21 or later
			ordered_var = pd.api.types.CategoricalDtype(ordered = True,
			categories = ordinal_var_dict[var])
			diamonds[var] = diamonds[var].astype(ordered_var)
		else: # pre-v0.21
			diamonds[var] = diamonds[var].astype('category', ordered = True,
			categories = ordinal_var_dict[var])
	
	
	# Compare Price vs. Carat
	np.random.seed(0)
	diamonds_sample = np.random.choice(diamonds.shape[0], 10000, replace = False)
	diamonds_subset = diamonds.loc[diamonds_sample]
	plt.scatter(data = diamonds_subset, x = 'carat', y = 'price', alpha = 1/3)
	plt.yscale('log')
	# See Figure 51
	
	# Still Price vs. Carat, but we'll use some more transformations
	diamonds['log_price'] = np.log(diamonds['price'])
	diamonds['cube_carat'] = diamonds['carat'] ** (1/3)
	plt.scatter(data = diamonds, x = 'cube_carat', y = 'log_price', alpha = 1/10)
	# See Figure 52
	
	# Let's compare price and the categorical variables
	"""
	sb.boxplot(data = diamonds, x = 'cut', y = 'price', color = base_color, order = cut_order)
	sb.boxplot(data = diamonds, x = 'color', y = 'price', color = base_color, order = color_order)
	sb.boxplot(data = diamonds, x = 'clarity', y = 'price', color = base_color, order = clarity_order)
	
	Which can all be done with the following commands:
	"""
	
	category_list = ['cut', 'color', 'clarity']
	non_category_list = diamonds.columns.difference(category_list)
	diamonds_c = diamonds.melt(id_vars = non_category_list, value_vars = category_list, var_name = 'main_category', value_name = 'main_value')

	sb.catplot(data = diamonds_c, y = 'price', x = 'main_value', col = 'main_category', kind = 'box', sharex= False, color = base_color)
	# See Figure 53
	
	"""
	Then, we'll check out the data after a transformation:
	
	sb.boxplot(data = diamonds, x = 'cut', y = 'log_price', color = base_color, order = cut_order)
	sb.boxplot(data = diamonds, x = 'color', y = 'log_price', color = base_color, order = color_order)
	sb.boxplot(data = diamonds, x = 'clarity', y = 'log_price', color = base_color, order = clarity_order)
	
	Which can all be done with the following commands:
	"""
	
	
	sb.catplot(data = diamonds_c, y = 'log_price', x = 'main_value', col = 'main_category', kind = 'box', sharex= False, color = base_color)
	# See Figure 54
	
	"""
	With a violin plot, you can get more insight into what causes the trend in median prices to appear as it does. Faceted histograms will also produce a similar result, though unless the faceting keeps the price axis common across facets, the trend will be harder to see. For each ordinal variable, there are multiple modes into which prices appear to fall. Going across increasing quality levels, you should see that the modes rise in price - this should be the expected effect of quality. However, you should also see that more of the data will be located in the lower-priced modes - this explains the unintuitive result noted in the previous comment. This is clearest in the clarity variable. Let's keep searching the data to see if there's more we can say about this pattern.
	"""
	
	"""
	
	Let's combine the boxplot images and the boxplot images using a combination of the methods we've seen:
	"""
	sb.violinplot(data = diamonds, x = 'cut', y = 'price', color = base_color, inner = None)
	sb.violinplot(data = diamonds, x = 'color', y = 'price', color = base_color, inner = None)
	sb.violinplot(data = diamonds, x = 'clarity', y = 'price', color = base_color, inner = None)
	
	sb.violinplot(data = diamonds, x = 'cut', y = 'carat', color = base_color, inner = None)
	sb.violinplot(data = diamonds, x = 'color', y = 'carat', color = base_color, inner = None)
	sb.violinplot(data = diamonds, x = 'clarity', y = 'carat', color = base_color, inner = None)
	"""
	
	
	
\end{python}

\begin{figure}
	\includegraphics{images/figure51.png}
	\caption{Price vs. Carat.}\label{fig:figure51}
\end{figure}

\begin{figure}
	\includegraphics{images/figure52.png}
	\caption{Price vs. Carat with Transformations.}\label{fig:figure52}
\end{figure}

\begin{figure}
	\includegraphics[width=\textwidth,height=\textheight,keepaspectratio]{images/figure53.png}
	\caption{Box Plots Price vs. Category.}\label{fig:figure53}
\end{figure}

\begin{figure}
	\includegraphics[width=\textwidth,height=\textheight,keepaspectratio]{images/figure54.png}
	\caption{Box Plots Price vs. Category.}\label{fig:figure54}
\end{figure}