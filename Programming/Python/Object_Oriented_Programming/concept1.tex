% concept1: Software Engineering Practices Pt. 1

\subsection{Learning Targets}

\begin{itemize}
	\item Write clean and modular mode
	\item Improve code efficiency
	\item Add effective documentation
	\item Use version control
\end{itemize}

\subsection{Clean and Modular Code}
\begin{itemize}
	\item Production code: software being used by live users
	\item Clean code: code that is readable, simple, and precise. Crucial for collaboration
	\item Modular code: code that is logically broken up into functions and modules
	\item Module: a file that allows code to be reused
\end{itemize}

\subsection{Refactoring Code}
\begin{itemize}
	\item Refactoring: restructuring the code to improve its internal structure without changing its external functionality
	\item Usually done after the code runs properly
	\item Easier to maintain, collaborate and reuse
\end{itemize}

\subsection{Tips on Writing Clean Code}
\begin{itemize}
	\item Use meaningful names
	\item Be descriptive and imply type
	\begin{itemize}
		\item Verbs for functions
		\item Nouns for variables
	\end{itemize}
	\item Use arbitrary naming conventions for variables in functions (i.e. a column of temperature data could be "temperature\_array")
	\item Use white space properly
\end{itemize}

\subsection{Writing Modular Code}
\begin{itemize}
	\item DRY: Don't Repeat Yourself
	\item Abstract out logic to improve readability
	\item Aim to minimize number of entities (functions, classes, modules, etc.), but also keep balance in mind
	\item Functions should do ONE thing
	\item Use arbitrary variable names
	\item Try for fewer (at most about 3) arguments per functions
\end{itemize}

\subsection{Efficient Code}
\begin{itemize}
	\item Reducing run time
	\item Reducing space in memory
	\item Tip: Use vector operations (vectorization) over loops hwne possible (think NumPy and Pandas functions)
\end{itemize}

\subsection{Documentation}
\begin{itemize}
	\item Clarify complex parts of code
	\item Navigate code easily
	\item Convey how and why code works
	\item Line Level: inline comments to clarify
	\item Function or Module Level: Docstrings to clarify
	\item Project Level: README to clarify
\end{itemize}

\subsection{Version Control (git)}
\begin{itemize}
	\item Scenario 1: Switching between module work, merging, and returning
	\item Scenario 2: Using detailed git commit lines, return to the faster running code and see what changed
	\item Scenario 3: Another user has been working on a separate branch while you were working on your branches
\end{itemize}
















