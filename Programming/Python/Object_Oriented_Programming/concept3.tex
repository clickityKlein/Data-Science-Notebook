% concept3: Introduction to Object-Oriented Programming

\subsection{Learning Targets}
\begin{itemize}
	\item Syntax of Object-Oriented Programming
	\begin{itemize}
		\item Procedural vs. Object-Oriented Programming
		\item Classes, Objects, Methods, and Attributes
		\item Coding a Class
		\item Magic Methods
		\item Inheritance
	\end{itemize}
	\item Build a Python Package that Analyzes Distributions
\end{itemize}

\subsection{Procedural vs. OOP}
\begin{itemize}
	\item Object: Specific instance of a class in which attributes (and sometimes methods) change
	\begin{itemize}
		\item Characteristics (attributes)
		\item Actions (methods)
	\end{itemize}
	\item Class: generic version of an object, almost a blueprint, which specifies attributes and methods
	\item Magic methods: overwrite default python behavior
\end{itemize}

\subsection{Inheritance}
\begin{itemize}
	\item Imagine a Clothing parent class which provides a blueprint for any type of clothing
	\item The general class helps to create more modular code and follow DRY
	\item Inheritance helps organize code with a more general version of a class which translates to more specified children classes
	\item Inheritance can make OOP more efficient to write
	\item Updates to a parent class automatically trickle down to its children!
\end{itemize}

\subsection{OOP Example: Clothing}
We'll be walking through the process of using object-oriented programming, starting with basic objects and then working our way to a generic parent class.
\\
Example 1: Create an object named Shirt
\begin{itemize}
	\item Attributes
	\begin{itemize}
		\item color
		\item size
		\item style
		\item price
	\end{itemize}
	\item Methods
	\begin{itemize}
		\item change\_price: set the price a the shirt to a new value
		\item discount: returns the price of a shirt with a discount applied
	\end{itemize}
\end{itemize}

\begin{python}
	class Shirt:
		def __init__(self, shirt_color, shirt_size, shirt_style, shirt_price):
			self.color = shirt_color
			self.size = shirt_size
			self.style = shirt_style
			self.price = shirt_price
		
		def change_price(self, new_price):
			self.price = new_price
		
		def discount(self, discount):
			return self.price * (1 - discount)
		
	"""
	Now, create a Shirt type object and test out the attributes and methods.
	"""
	new_shirt = Shirt('red', 'S', 'short sleeve', 15)
	
	print(new_shirt.color)
	print(new_shirt.size)
	print(new_shirt.style)
	print(new_shirt.price)
	"""
	red
	S
	short sleeve
	15
	"""
	
	new_shirt.change_price(10)
	print(new_shirt.price)
	# 10
	
	print(new_shirt.discount(0.2))
	# 8.0
\end{python}

Similar to the above, write a Pants class. This class will have waist size and length instead of shirt size and style, respectively.

\begin{python}
	class Pants:        
		def __init__(self, color, waist_size, length, price):
			"""Method for initializing a Pants object
			
			Args: 
			color (str)
			waist_size (int)
			length (int)
			price (float)
			
			Attributes:
			color (str): color of a pants object
			waist_size (str): waist size of a pants object
			length (str): length of a pants object
			price (float): price of a pants object
			"""
			
			self.color = color
			self.waist_size = waist_size
			self.length = length
			self.price = price
		
		def change_price(self, new_price):
			"""The change_price method changes the price attribute of a pants object
			
			Args: 
			new_price (float): the new price of the pants object
			
			Returns: None
			
			"""
			self.price = new_price
		
		def discount(self, discount):
			"""The discount method outputs a discounted price of a pants object
			
			Args:
			percentage (float): a decimal representing the amount to discount
			
			Returns:
			float: the discounted price
			"""
			return self.price * (1 - discount)
\end{python}

Now, let's see how we can combine and use the objects we create. Make a SalesPerson class with information and methods pertaining to selling pants.
\begin{python}
	class SalesPerson:
		def __init__(self, first_name, last_name, employee_id, salary):
			self.first_name = first_name
			self.last_name = last_name
			self.employee_id = employee_id
			self.salary = salary
			self.pants_sold = []
			self.total_sales = 0
		
		def sell_pants(self, pants_object):
			self.pants_sold.append(pants_object)
		
		def calculate_sales(self):
			self.total_sales = sum([pants.price for pants in self.pants_sold])
			return self.total_sales
		
		def display_sales(self):
			for pants in self.pants_sold:
			print(f'color: {pants.color}, length: {pants.length}, price: {pants.price}, waist size: {pants.waist_size}')
		
		def calculate_commission(self, commission):
			sales_total = self.calculate_sales()
			return sales_total * commission
\end{python}

Something notable about SalesPerson class is the fact we set attributes to predefined values, such as the blank list and value of 0. Additionally, note that we were able to call Pants class attributes after they had been passed in.
\\

Moving into the more general parent class we discussed while describing inheritance, let's create that general Clothing class.

\begin{python}
	class Clothing:
		def __init__(self, color, size, style, price):
			self.color = color
			self.size = size
			self.style = style
			self.price = price
		
		def change_price(self, price):
			self.price = price
		
		def calculate_discount(self, discount):
			return self.price * (1 - discount)
		
		def calculate_shipping(self, weight, rate):
			return weight * rate
	"""
	Children classes will inherit Clothing's attributes and methods,
	but can also be added to and overwritten.
	See Below for some examples:
	"""
	
	# Shirt
	class Shirt(Clothing):
		def __init__(self, color, size, style, price, long_or_short):
			Clothing.__init__(self, color, size, style, price)
			self.long_or_short = long_or_short
		
		def double_price(self):
		self.price = 2 * self.price
	
	# Pants
	class Pants(Clothing):
		def __init__(self, color, size, style, price, waist):
			Clothing.__init__(self, color, size, style, price)
			self.waist = waist
		
		def calculate_discount(self, discount):
			return self.price * (discount / 2)
	
	# Blouse
	class Blouse(Clothing):
		def __init__(self, color, size, style, price, country_of_origin):
			Clothing.__init__(self, color, size, style, price)
			self.country_of_origin = country_of_origin
		
		def triple_price(self):
			return self.price * 3
\end{python}


\subsection{OOP Example: Distributions}


\subsection{Uploading a Package to PyPI}
\begin{itemize}
	\item Recommended to create a virtual environment (almost like a silo) to test installing packages (won't affect main Python installation)
	\item Virtual environments:
	\begin{itemize}
		\item Conda: manages packages, manages environments
		\item conda create --name environmentname
		\item source activate environmentname
		\item conda install numpy
		\item Pip and Venv: Pip is a package manager, Venv is an environment manager that comes preinstalled with Python 3
		\item Recommended to create a Conda environment and install Pip simultaneously
		\begin{itemize}
			\item conda create -–name environmentname pip
		\end{itemize}
		\item Pip with Venv work as expected, used for generic software development projects including web development
		\begin{itemize}
			\item python -m venv venv\_name
			\item source venv\_name/bin/activate (activates virtual environment)
		\end{itemize}
	\end{itemize}
\end{itemize}










