% concept2: Software Engineering Practices Pt. 2

\subsection{Learning Targets}
\begin{itemize}
	\item Testing
	\item Logging
	\item Code Reviews
\end{itemize}

\subsection{Testing}
\begin{itemize}
	\item Essential before deployment
	\item Many data scientist's code goes into production without testing, and can be an annoyance to the software developers
	\item Test Driven Development (TDD): development process in which tests are written prior to code, itself
	\item Unit Test: a test covering a unit of code, usually a single function
\end{itemize}

\subsection{Unit Tests Basics}
\begin{itemize}
	\item Useful when testing hundreds of functions repeatedly while tracking the outcomes
	\item Reduces manual tests
	\item Repeatable and Automated!
	\item Isolated from the rest of program, thus no dependencies involved
	\item A renowned testing tool: pytest
	\item TDD: write unit tests as programming for errors that are likely occur
\end{itemize}

\subsection{Logging}
\begin{itemize}
	\item Logging: Recording of Errors / Error Messages
	\item Be professional and clear
	\item Be concise and use normal capitalization
	\item Choose appropriate level for logging:
	\begin{itemize}
		\item Debug: use this "level" for anything that happens in the program
		\item Error: use for any errors that occur
		\item Info: records all actions that are user-driven or system-specific (regularly scheduled operations)
		\item Provide any useful information
	\end{itemize}
\end{itemize}

\subsection{Code Reviews}
\begin{itemize}
	\item Catch errors
	\item Ensure readability
	\item Check standards are met
	\item Share knowledge (personal, professional, team, etc.)
	\item Questions to ask when conducting a code review:
	\begin{itemize}
		\item Can I understand the code easily?
		\item Does it use meaningful names and whitespace?
		\item Is there duplicated code?
		\item Can I provide another layer of abstraction?
		\item Is each function and module necessary?
		\item Is each function or module the right length?
	\end{itemize}
	\item Is the code efficient?
	\begin{itemize}
		\item Are there loops or other steps I can vectorize?
		\item Can I use better data structures to optimize any steps?
		\item Can I use generators or multiprocessing to optimize any steps?
	\end{itemize}
	\item Is the documentation effective?
	\begin{itemize}
		\item Are inline comments concise and meaningful?
		\item Is there complex code that missing documentation?
		\item Do functions use effective docstrings?
		\item Is the necessary project documentation provided?
	\end{itemize}
	\item Is the code well tested?
	\begin{itemize}
		\item Does the code have test coverage?
		\item Do tests check for interesting cases?
		\item Are the tests readable?
		\item Can the tests be made more efficient?
	\end{itemize}
	\item Is the logging effective?
	\begin{itemize}
		\item Are the log messages clear, concise, and professional?
		\item Do they include all relevant and useful information?
		\item Do they use the appropriate logging level?
	\end{itemize}
	\item Tips for code review:
	\begin{itemize}
		\item Use a code linter: linters like pylint can check for coding standard and PEP8 guidelines
		\item Agree on a style guide as a team
		\item Explain issues and make suggestions
		\item Keep comments objective (avoid "I" and "you")
	\end{itemize}
\end{itemize}