% concept3: Introduction to Object-Oriented Programming

\subsection{Learning Targets}
\begin{itemize}
	\item Syntax of Object-Oriented Programming
	\begin{itemize}
		\item Procedural vs. Object-Oriented Programming
		\item Classes, Objects, Methods, and Attributes
		\item Coding a Class
		\item Magic Methods
		\item Inheritance
	\end{itemize}
	\item Build a Python Package that Analyzes Distributions
\end{itemize}

\subsection{Procedural vs. OOP}
\begin{itemize}
	\item Object: Specific instance of a class in which attributes (and sometimes methods) change
	\begin{itemize}
		\item Characteristics (attributes)
		\item Actions (methods)
	\end{itemize}
	\item Class: generic version of an object, almost a blueprint, which specifies attributes and methods
	\item Magic methods: overwrite default python behavior
\end{itemize}

\subsection{Inheritance}
\begin{itemize}
	\item Imagine a Clothing parent class which provides a blueprint for any type of clothing
	\item The general class helps to create more modular code and follow DRY
	\item Inheritance helps organize code with a more general version of a class which translates to more specified children classes
	\item Inheritance can make OOP more efficient to write
	\item Updates to a parent class automatically trickle down to its children!
\end{itemize}

\subsection{OOP Example: Clothing}
We'll be walking through the process of using object-oriented programming, starting with basic objects and then working our way to a generic parent class.
\\
Example 1: Create an object named Shirt
\begin{itemize}
	\item Attributes
	\begin{itemize}
		\item color
		\item size
		\item style
		\item price
	\end{itemize}
	\item Methods
	\begin{itemize}
		\item change\_price: set the price a the shirt to a new value
		\item discount: returns the price of a shirt with a discount applied
	\end{itemize}
\end{itemize}

\begin{python}
	class Shirt:
		def __init__(self, shirt_color, shirt_size, shirt_style, shirt_price):
			self.color = shirt_color
			self.size = shirt_size
			self.style = shirt_style
			self.price = shirt_price
		
		def change_price(self, new_price):
			self.price = new_price
		
		def discount(self, discount):
			return self.price * (1 - discount)
		
	"""
	Now, create a Shirt type object and test out the attributes and methods.
	"""
	new_shirt = Shirt('red', 'S', 'short sleeve', 15)
	
	print(new_shirt.color)
	print(new_shirt.size)
	print(new_shirt.style)
	print(new_shirt.price)
	"""
	red
	S
	short sleeve
	15
	"""
	
	new_shirt.change_price(10)
	print(new_shirt.price)
	# 10
	
	print(new_shirt.discount(0.2))
	# 8.0
\end{python}

Similar to the above, write a Pants class. This class will have waist size and length instead of shirt size and style, respectively.

\begin{python}
	class Pants:        
		def __init__(self, color, waist_size, length, price):
			"""Method for initializing a Pants object
			
			Args: 
			color (str)
			waist_size (int)
			length (int)
			price (float)
			
			Attributes:
			color (str): color of a pants object
			waist_size (str): waist size of a pants object
			length (str): length of a pants object
			price (float): price of a pants object
			"""
			
			self.color = color
			self.waist_size = waist_size
			self.length = length
			self.price = price
		
		def change_price(self, new_price):
			"""The change_price method changes the price attribute of a pants object
			
			Args: 
			new_price (float): the new price of the pants object
			
			Returns: None
			
			"""
			self.price = new_price
		
		def discount(self, discount):
			"""The discount method outputs a discounted price of a pants object
			
			Args:
			percentage (float): a decimal representing the amount to discount
			
			Returns:
			float: the discounted price
			"""
			return self.price * (1 - discount)
\end{python}

Now, let's see how we can combine and use the objects we create. Make a SalesPerson class with information and methods pertaining to selling pants.
\begin{python}
	class SalesPerson:
		def __init__(self, first_name, last_name, employee_id, salary):
			self.first_name = first_name
			self.last_name = last_name
			self.employee_id = employee_id
			self.salary = salary
			self.pants_sold = []
			self.total_sales = 0
		
		def sell_pants(self, pants_object):
			self.pants_sold.append(pants_object)
		
		def calculate_sales(self):
			self.total_sales = sum([pants.price for pants in self.pants_sold])
			return self.total_sales
		
		def display_sales(self):
			for pants in self.pants_sold:
			print(f'color: {pants.color}, length: {pants.length}, price: {pants.price}, waist size: {pants.waist_size}')
		
		def calculate_commission(self, commission):
			sales_total = self.calculate_sales()
			return sales_total * commission
\end{python}

Something notable about SalesPerson class is the fact we set attributes to predefined values, such as the blank list and value of 0. Additionally, note that we were able to call Pants class attributes after they had been passed in.
\\

Moving into the more general parent class we discussed while describing inheritance, let's create that general Clothing class.

\begin{python}
	class Clothing:
		def __init__(self, color, size, style, price):
			self.color = color
			self.size = size
			self.style = style
			self.price = price
		
		def change_price(self, price):
			self.price = price
		
		def calculate_discount(self, discount):
			return self.price * (1 - discount)
		
		def calculate_shipping(self, weight, rate):
			return weight * rate
	"""
	Children classes will inherit Clothing's attributes and methods,
	but can also be added to and overwritten.
	See Below for some examples:
	"""
	
	# Shirt
	class Shirt(Clothing):
		def __init__(self, color, size, style, price, long_or_short):
			Clothing.__init__(self, color, size, style, price)
			self.long_or_short = long_or_short
		
		def double_price(self):
		self.price = 2 * self.price
	
	# Pants
	class Pants(Clothing):
		def __init__(self, color, size, style, price, waist):
			Clothing.__init__(self, color, size, style, price)
			self.waist = waist
		
		def calculate_discount(self, discount):
			return self.price * (discount / 2)
	
	# Blouse
	class Blouse(Clothing):
		def __init__(self, color, size, style, price, country_of_origin):
			Clothing.__init__(self, color, size, style, price)
			self.country_of_origin = country_of_origin
		
		def triple_price(self):
			return self.price * 3
\end{python}


\subsection{OOP Example: Distributions}
We'll sidestep the learning order and exploration phase somewhat in these notes. It wouldn't be unusual to build out a specific class for a distribution, and then given a more diverse need, build out a parent class for the distributions. However, we'll present a parent Distribution class, and then build a few child distributions through inheritance.
\\

\begin{python}
	class Distribution:
		def __init__(self, mu=0, sigma=1):
		""" Generic distribution class for calculating and 
		visualizing a probability distribution.
		
			Attributes:
			mean (float) representing the mean value of the distribution
			stdev (float) representing the standard deviation of the distribution
			data_list (list of floats) a list of floats extracted from the data file
			"""
			self.mean = mu
			self.stdev = sigma
			self.data = []
		
		def read_data_file(self, file_name):
		
			"""Function to read in data from a txt file. The txt file should have
			one number (float) per line. The numbers are stored in the data 	attribute.
			
			Args:
			file_name (string): name of a file to read from
			
			Returns:
			None
			
			"""
			
			with open(file_name) as file:
			data_list = []
			line = file.readline()
			while line:
			data_list.append(int(line))
			line = file.readline()
			file.close()
			
			self.data = data_list
\end{python}   

Note a new method with this class is the use of reading in an outside file. It may be useful to consider files methods that will read in or output files of different types.
\\

\begin{python}
	class Gaussian(Distribution):
		""" Gaussian distribution class for calculating and 
		visualizing a Gaussian distribution.
		
		Attributes:
		mean (float) representing the mean value of the distribution
		stdev (float) representing the standard deviation of the distribution
		data_list (list of floats) a list of floats extracted from the data file
		
		"""
		def __init__(self, mu=0, sigma=1):
			Distribution.__init__(self, mu, sigma)
		
		def calculate_mean(self):
			"""Function to calculate the mean of the data set.
			
			Args: 
			None
			
			Returns: 
			float: mean of the data set
			
			"""
			avg = 1.0 * sum(self.data) / len(self.data)
			
			self.mean = avg
			
			return self.mean
		
		def calculate_stdev(self, sample=True):
			"""Function to calculate the standard deviation of the data set.
			
			Args: 
			sample (bool): whether the data represents a sample or population
			
			Returns: 
			float: standard deviation of the data set
			
			"""
			
			if sample:
			n = len(self.data) - 1
			else:
			n = len(self.data)
			
			mean = self.calculate_mean()
			
			sigma = 0
			
			for d in self.data:
			sigma += (d - mean) ** 2
			
			sigma = math.sqrt(sigma / n)
			
			self.stdev = sigma
			
			return self.stdev
		
		def plot_histogram(self):
			"""Function to output a histogram of the instance variable data using 
			matplotlib pyplot library.
			
			Args:
			None
			
			Returns:
			None
			"""
			plt.hist(self.data)
			plt.title('Histogram of Data')
			plt.xlabel('data')
			plt.ylabel('count')
		
		def pdf(self, x):
			"""Probability density function calculator for the gaussian distribution.
			
			Args:
			x (float): point for calculating the probability density function
			
			
			Returns:
			float: probability density function output
			"""
			
			return (1.0 / (self.stdev * math.sqrt(2*math.pi))) * math.exp(-0.5*((x - self.mean) / self.stdev) ** 2)
		
		def plot_histogram_pdf(self, n_spaces = 50):
			"""Function to plot the normalized histogram of the data and a plot of 	the 
			probability density function along the same range
			
			Args:
			n_spaces (int): number of data points 
			
			Returns:
			list: x values for the pdf plot
			list: y values for the pdf plot
			
			"""
			
			mu = self.mean
			sigma = self.stdev
			
			min_range = min(self.data)
			max_range = max(self.data)
			
			# calculates the interval between x values
			interval = 1.0 * (max_range - min_range) / n_spaces
			
			x = []
			y = []
			
			# calculate the x values to visualize
			for i in range(n_spaces):
			tmp = min_range + interval*i
			x.append(tmp)
			y.append(self.pdf(tmp))
			
			# make the plots
			fig, axes = plt.subplots(2,sharex=True)
			fig.subplots_adjust(hspace=.5)
			axes[0].hist(self.data, density=True)
			axes[0].set_title('Normed Histogram of Data')
			axes[0].set_ylabel('Density')
			
			axes[1].plot(x, y)
			axes[1].set_title('Normal Distribution for \n Sample Mean and Sample Standard Deviation')
			axes[0].set_ylabel('Density')
			plt.show()
			
			return x, y
		
		def __add__(self, other):
			"""Function to add together two Gaussian distributions
			
			Args:
			other (Gaussian): Gaussian instance
			
			Returns:
			Gaussian: Gaussian distribution
			
			"""
			result = Gaussian()
			result.mean = self.mean + other.mean
			result.stdev = math.sqrt(self.stdev ** 2 + other.stdev ** 2)
			
			return result
		
		def __repr__(self):
			"""Function to output the characteristics of the Gaussian instance
			
			Args:
			None
			
			Returns:
			string: characteristics of the Gaussian
			
			"""
			
			return "mean {}, standard deviation {}".format(self.mean, self.stdev)
\end{python}

Note that in the build of child Gaussian class we feature a method to produce a visual, and we introduce our first magic methods.
\\

Magic methods override the main functionality of core Python methods, and have the telltale sign of the double underscores on each side (dunders, if you will).
\\

Next up, we add more variability with the Binomial distribution.

\begin{python}
	class Binomial(Distribution):
		""" Binomial distribution class for calculating and 
		visualizing a Binomial distribution.
		
		Attributes:
		mean (float) representing the mean value of the distribution
		stdev (float) representing the standard deviation of the distribution
		data_list (list of floats) a list of floats to be extracted from the data 	file
		p (float) representing the probability of an event occurring
		n (int) the total number of trials
		
		
		TODO: Fill out all TODOs in the functions below
		
		"""
		
		#       A binomial distribution is defined by two variables: 
		#           the probability of getting a positive outcome
		#           the number of trials
		
		#       If you know these two values, you can calculate the mean and the 	standard deviation
		#       
		#       For example, if you flip a fair coin 25 times, p = 0.5 and n = 25
		#       You can then calculate the mean and standard deviation with the 	following formula:
		#           mean = p * n
		#           standard deviation = sqrt(n * p * (1 - p))
		
		#       
		
		def __init__(self, mu = 0, sigma = 1, prob = 0.5, size = 20):
			Distribution.__init__(self, mu, sigma)
			self.prob = prob
			self.size = size
		
		# TODO: store the probability of the distribution in an instance variable p
		# TODO: store the size of the distribution in an instance variable n
		
		# TODO: Now that you know p and n, you can calculate the mean and standard 	deviation
		#       Use the calculate_mean() and calculate_stdev() methods to calculate 	the
		#       distribution mean and standard deviation
		#
		#       Then use the init function from the Distribution class to 	initialize the
		#       mean and the standard deviation of the distribution
		#
		#       Hint: You need to define the calculate_mean() and calculate_stdev() 	methods
		#               farther down in the code starting in line 55. 
		#               The init function can get access to these methods via the 	self
		#               variable.             
		
		def calculate_mean(self):
			"""Function to calculate the mean from p and n
			
			Args: 
			None
			
			Returns: 
			float: mean of the data set
			
			"""
			
			# TODO: calculate the mean of the Binomial distribution. Store the mean
			#       via the self variable and also return the new mean value
			
			self.mu = self.prob * self.size
			return self.mu
		
		def calculate_stdev(self):
			"""Function to calculate the standard deviation from p and n.
			
			Args: 
			None
			
			Returns: 
			float: standard deviation of the data set
			
			"""
			# TODO: calculate the standard deviation of the Binomial distribution. 	Store
			#       the result in the self standard deviation attribute. Return the 	value
			#       of the standard deviation.
			# sqrt(n * p * (1 - p))
			self.sigma = math.sqrt(self.size * self.prob * (1 - self.prob))
			return self.sigma
		
		def replace_stats_with_data(self):
			"""Function to calculate p and n from the data set
			
			Args: 
			None
			
			Returns: 
			float: the p value
			float: the n value
			
			"""
			# TODO: The read_data_file() from the Generaldistribution class can 	read in a data
			#       file. Because the Binomaildistribution class inherits from the 	Generaldistribution class,
			#       you don't need to re-write this method. However,  the method
			#       doesn't update the mean or standard deviation of
			#       a distribution. Hence you are going to write a method that 	calculates n, p, mean and
			#       standard deviation from a data set and then updates the n, p, 	mean and stdev attributes.
			#       Assume that the data is a list of zeros and ones like [0 1 0 1 	1 0 1]. 
			#
			#       Write code that: 
			#           updates the n attribute of the binomial distribution
			#           updates the p value of the binomial distribution by 	calculating the
			#               number of positive trials divided by the total trials
			#           updates the mean attribute
			#           updates the standard deviation attribute
			#
			#       Hint: You can use the calculate_mean() and calculate_stdev() 	methods
			#           defined previously.
			self.size = len(self.data)
			self.calculate_mean()
			self.calculate_stdev()
			
		def plot_bar(self):
			"""Function to output a histogram of the instance variable data using 
			matplotlib pyplot library.
			
			Args:
			None
			
			Returns:
			None
			"""
			
			# TODO: Use the matplotlib package to plot a bar chart of the data
			#       The x-axis should have the value zero or one
			#       The y-axis should have the count of results for each case
			#
			#       For example, say you have a coin where heads = 1 and tails = 0.
			#       If you flipped a coin 35 times, and the coin landed on
			#       heads 20 times and tails 15 times, the bar chart would have two 	bars:
			#       0 on the x-axis and 15 on the y-axis
			#       1 on the x-axis and 20 on the y-axis
			#       Make sure to label the chart with a title, x-axis label and 	y-axis label
			dat = np.array(self.data)
			zero = len(dat[dat==0])
			one = len(dat[dat==1])
			plt.bar(plt.bar(['zero','one'],[zero, one]))
			plt.xlabel('Result')
			plt.ylabel('Count')
			plt.title('Counts of Binomial Data')
		
		def pdf(self, k):
			"""Probability density function calculator for the gaussian 	distribution.
			
			Args:
			k (float): point for calculating the probability density function
			
			Returns:
			float: probability density function output
			"""
			# TODO: Calculate the probability density function for a binomial 	distribution
			#  For a binomial distribution with n trials and probability p, 
			#  the probability density function calculates the likelihood of getting
			#   k positive outcomes. 
			# 
			#   For example, if you flip a coin n = 60 times, with p = .5,
			#   what's the likelihood that the coin lands on heads 40 out of 60 	times?
			n_choose_k = math.factorial(self.size) / (math.factorial(k) * 	(math.factorial(self.size - k)))
			test_prob = n_choose_k * (self.prob ** k) * ((1 - self.prob) ** 	(self.size - k))
			return test_prob
		
		def plot_bar_pdf(self):
			"""Function to plot the pdf of the binomial distribution
			
			Args:
			None
			
			Returns:
			list: x values for the pdf plot
			list: y values for the pdf plot
			"""
			# TODO: Use a bar chart to plot the probability density function from
			# k = 0 to k = n
			
			#   Hint: You'll need to use the pdf() method defined above to 	calculate the
			#   density function for every value of k.
			
			#   Be sure to label the bar chart with a title, x label and y label
			
			#   This method should also return the x and y values used to make the 	chart
			#   The x and y values should be stored in separate lists
			k_values = np.arange(self.size + 1)
			k_probs = [self.pdf(k) for k in k_values]
			plt.bar(k_values, k_probs)
			plt.xlabel('Successful Outcomes')
			plt.ylabel('Probability')
			plt.title('Probability  of Successful Outcomes')
		
		def __add__(self, other):
			"""Function to add together two Binomial distributions with equal p
			
			Args:
			other (Binomial): Binomial instance
			
			Returns:
			Binomial: Binomial distribution
			
			"""
			try:
			assert self.p == other.p, 'p values are not equal'
			except AssertionError as error:
			raise
			
			# TODO: Define addition for two binomial distributions. Assume that the
			# p values of the two distributions are the same. The formula for 
			# summing two binomial distributions with different p values is more 	complicated,
			# so you are only expected to implement the case for two distributions with equal p.
			
			# the try, except statement above will raise an exception if the p 	values are not equal
			
			# Hint: You need to instantiate a new binomial object with the correct 	n, p, 
			#   mean and standard deviation values. The __add__ method should 	return this
			#   new binomial object.
			
			#   When adding two binomial distributions, the p value remains the same
			#   The new n value is the sum of the n values of the two distributions.
			new_distr = Binomial()
			new_distr.prob = self.prob
			new_distr.size = self.size + other.size
			new_distr.calculate_mean()
			new_distr.calculate_stdev()
		
		def __repr__(self):
		
			"""Function to output the characteristics of the Binomial instance
			
			Args:
			None
			
			Returns:
			string: characteristics of the Gaussian
			
			"""
			# TODO: Define the representation method so that the output looks like
			#       mean 5, standard deviation 4.5, p .8, n 20
			#
			#       with the values replaced by whatever the actual distributions 	values are
			#       The method should return a string in the expected format
			return f'mean {self.mu}, standard deviation {self.stdev}, p 	{self.prob}, n {self.size}'
\end{python}


\subsection{Uploading a Package to PyPI}
\begin{itemize}
	\item Recommended to create a virtual environment (almost like a silo) to test installing packages (won't affect main Python installation)
	\item Virtual environments:
	\begin{itemize}
		\item Conda: manages packages, manages environments
		\item conda create --name environmentname
		\item source activate environmentname
		\item conda install numpy
		\item Pip and Venv: Pip is a package manager, Venv is an environment manager that comes preinstalled with Python 3
		\item Recommended to create a Conda environment and install Pip simultaneously
		\begin{itemize}
			\item conda create -–name environmentname pip
		\end{itemize}
		\item Pip with Venv work as expected, used for generic software development projects including web development
		\begin{itemize}
			\item python -m venv venv\_name
			\item source venv\_name/bin/activate (activates virtual environment)
		\end{itemize}
	\end{itemize}
\end{itemize}










