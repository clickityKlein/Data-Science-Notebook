% Boolean Indexing, Set Operations, and Sorting
Now that we have a solid understanding of elements and slicing in rank 1 and rank 2 arrays, we can move into conditionals and comparing arrays with each other. Some essentials covered in this section:

\begin{itemize}
	\item intersect1d: this will find the intersection of two arrays and return the sorted, unique values
	\item setdiff1d: this will find the differences between two arrays and return the sorted, unique values
	\item union1d: this will find the union between two arrays and return the sorted, unique values
	\item sort: this will sort the values of an array 
\end{itemize}

\begin{python}
	X = np.arange(25).reshape(5, 5)
	"""
	[[ 0  1  2  3  4]
	[ 5  6  7  8  9]
	[10 11 12 13 14]
	[15 16 17 18 19]
	[20 21 22 23 24]]
	"""
	
	# first look at conditional arguments
	X[X > 10]
	X[(X > 10) & (X < 17)]
	"""
	[11 12 13 14 15 16 17 18 19 20 21 22 23 24]
	[11 12 13 14 15 16]
	"""
	
	# basic application of conditional arguments (other than slicing)
	X[(X > 10) & (X < 17)] = -1
	"""
	[[ 0  1  2  3  4]
	[ 5  6  7  8  9]
	[10 -1 -1 -1 -1]
	[-1 -1 17 18 19]
	[20 21 22 23 24]]
	"""
	
	x = np.array([1, 2, 3, 4])
	y = np.array([3, 4, 5, 6])
	np.intersect1d(x, y)
	np.setdiff1d(x, y)
	np.union1d(x, y)
	"""
	Recall these functions return sorted, unique arrays
	
	Intersection
	[3 4]
	
	Difference
	[1 2]
	
	Union
	[1 2 3 4 5 6]
	"""
	x = np.random.randint(1, 11, size = (10,))
	np.sort(x)
	"""
	np.sort(x) =
	[2 2 2 2 3 5 5 5 7 9]
	
	x =
	[2 5 2 5 5 3 2 9 7 2]
	"""
	
	np.sort(np.unique(x))
	"""
	[2 3 5 7 9]
	"""
	
	# can also sort using x.sort()
	x.sort()
	"""
	x =
	[2 2 2 2 3 5 5 5 7 9]
	
	Note the difference between np.sort(x) and x.sort():
	- np.sort(x) creates an instance where x is sorted
	- x.sort() changes the array to be sorted
	"""
	
	x = np.random.randint(1, 11, size = (5,5))
	"""
	[[ 9  2 10  6  6]
	[10  7  6 10  8]
	[ 3  1  1  4  2]
	[ 9  6  5  7  3]
	[10  7  5  7  7]]
	"""
	
	# can sort 2d arrays using the axis command
	# sort rows with axis = 0, columns with axis = 1
	np.sort(x, axis = 0)
	np.sort(x, axis = 1)
	"""
	axis = 0 (row sort)
	[[ 3  1  1  4  2]
	[ 9  2  5  6  3]
	[ 9  6  5  7  6]
	[10  7  6  7  7]
	[10  7 10 10  8]]
	
	axis = 1 (column sort)
	[[ 2  6  6  9 10]
	[ 6  7  8 10 10]
	[ 1  1  2  3  4]
	[ 3  5  6  7  9]
	[ 5  7  7  7 10]]
	"""
\end{python}

It's important to note that we should be careful and deliberate when using the sort function. For instance, say we have a rank 2 array that is comprised of data entries where either each row or each column are data points referring to a specific event. If not done properly, sorting could mix the data from the events.