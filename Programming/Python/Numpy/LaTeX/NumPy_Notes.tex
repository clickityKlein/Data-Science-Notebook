\documentclass{article}
\usepackage{listings}
\usepackage[colorlinks=true,linkcolor=black,anchorcolor=black,citecolor=black,filecolor=black,menucolor=black,runcolor=black,urlcolor=black]{hyperref}
\usepackage{pythonhighlight}
\usepackage{graphicx}
\setlength{\parindent}{0pt}

\begin{document}
	\title{Python: NumPy Course Notes}
	\author{Klein \\ carlj.klein@gmail.com}
	\date{}
	\maketitle
	
	\section{Learning Objectives}
	Concepts
	\begin{itemize}
		\item \nameref{sec:concept1}
		\item \nameref{sec:concept2}
		\item \nameref{sec:concept3}
		\item \nameref{sec:concept4}
		\item \nameref{sec:concept5}
		\item \nameref{sec:exercise1}
		\item \nameref{sec:concept6}
		\item \nameref{sec:exercise2}
	\end{itemize}
	Commands
	\begin{itemize}
		\item np.array
		\item .shape
		\item .size
		\item np.save and np.load
		\item np.zeros
		\item np.ones
		\item np.eye
		\item np.diag
		\item np.arange
		\item np.linspace
		\item np.reshape
		\item np.random
			\begin{itemize}
				\item np.random.random
				\item np.random.randint
				\item np.random.normal
				\item np.random.permutation
			\end{itemize}
		\item np.delete
		\item np.append
		\item np.insert
		\item np.vstack
		\item np.hstack
		\item np.unique
		\item np.copy
		\item np.intersect1d
		\item np.setdiff1d
		\item np.union1d
		\item np.sort
		\item Mathematical Functions
		\begin{itemize}
			\item np.add
			\item np.subtract
			\item np.multiply
			\item np.divide
			\item np.sqrt
			\item np.exp
			\item np.power 
		\end{itemize}
		\item Statistical Functions
		\begin{itemize}
			\item mean
			\item std
			\item median
			\item max
			\item min
		\end{itemize}
	\end{itemize}

NumPy is useful in dealing with arrays of data, which can be thought of in a mathematical sense as vectors and matrices. NumPy is a powerful tool in itself, however an even more useful Python data tool, the Pandas library, is built on top of NumPy. A basic understanding of NumPy is necessary, and we'll cover some of the essential functions and uses in these notes.  

\section{Creating and Saving NumPy ndarrays}\label{sec:concept1}
% Creating and Saving NumPy ndarrays

Not needing too much of an explanation, here's a beginning look at creating basic NumPy arrays.
\\\\
Covered in this section:
\begin{itemize}
	\item np.array: turn a list, or list of lists, into a NumPy array object
	\item shape: return the dimensions of an array object
	\item size: return the total number of elements in an array object
	\item save \& load: arrays can be saved for future use
\end{itemize}

\begin{python}
	x = np.array([1 ,2, 3, 4, 5])
	print(x)
	# [1 2 3 4 5]
	
	x.shape
	# (5,)
	
	Y = np.array([[1, 2, 3], [4, 5, 6], [7, 8, 9], [10, 11, 12]])
	print(Y)
	"""
	[[ 1  2  3]
	[ 4  5  6]
	[ 7  8  9]
	[10 11 12]]
	"""
	
	Y.shape
	# (4, 3)
	Y.size
	# 12
	
	# Notes Section:
	# dtype changes with mixed data, and will uptype (int + floats -> all floats) 
	# numpy arrays can be saved for later use
	
	# save
	np.save('my_array', x)
	# load
	y = np.load('my_array.npy')
\end{python}


\section{Using Built-in Functions to Create ndarrays}\label{sec:concept2}
% Using Built-in Functions to Create ndarrays

\section{Accessing, Deleting, and Inserting Elements Into ndarrays}\label{sec:concept3}
% Concept3: Explanatory Visuals

\section{Slicing ndarrays}\label{sec:concept4}
% Concept4: Visualization Case Study

\section{Boolean Indexing, Set Operations, and Sorting}\label{sec:concept5}

\section{Exercise: Manipulating ndarrays}\label{sec:exercise1}

\section{Arithmetic Operations and Broadcasting}\label{sec:concept6}

\section{Exercise: Creating ndarrays with Broadcasting}\label{sec:exercise2}

\end{document}