% Slicing ndarrays
Covered briefly in the previous section, we're going to do a deeper dive into the slicing capabilities of NumPy. Aside from syntax, some essential functions covered in this section are:

\begin{itemize}
	\item copy: this creates an entirely new array, where changes won't be mapped back to the original
	\item diag: another use of the diagonal function, this is used to return an array featuring the diagonal of a rank 2 array
	\item unique: this will find all of the unique values within an array
\end{itemize}

\begin{python}
	# Here are the basic ways to slice a rank 1 array:
	# 1. ndarray[start:end]
	# 2. ndarray[start:]
	# 3. ndarray[:end]
	# Also, recall that ndarray[-1] will return the last element in an array
	
	# examples of slicing in rank 2 array
	X = np.arange(1, 21).reshape(4, 5)
	"""
	[[ 1  2  3  4  5]
	[ 6  7  8  9 10]
	[11 12 13 14 15]
	[16 17 18 19 20]]
	"""
	
	# the rank 2 slice order is: rows, columns (starts are inclusive, ends are exclusive)
	z = X[1:4, 2:5]
	"""
	[[ 8  9 10]
	[13 14 15]
	[18 19 20]]
	"""
	
	# by leaving either the row or column input as ":" it will select entire row or column
	# this will return the 3rd column
	z = X[:, 2]
	# [ 3  8 13 18]
	# however, that returns a rank 1 array (row format). Let's return the correct dimensions:
	z = X[:, 2:3]
	"""
	[[ 3]
	[ 8]
	[13]
	[18]]
	"""
	
	# slicing creates a "view" of the array
	# changing z will change X (keep the copy command in mind here)
	z = X[1:, 2:]
	z[2, 2] = 0
	"""
	X =
	[[ 1  2  3  4  5]
	[ 6  7  8  9 10]
	[11 12 13 14 15]
	[16 17 18 19  0]]
	
	Even though the elements were changed in the z array, there was still a connection to the X array.
	"""
	
	# create a brand new array (so to not change the original array)
	z = X[1:, 2:].copy()
	z[2, 2] = 0
	"""
	X =
	[[ 1  2  3  4  5]
	[ 6  7  8  9 10]
	[11 12 13 14 15]
	[16 17 18 19 20]]
	
	By using the copy function the connection to the original array was broken.
	"""
	
	# use variable in slicing
	indices = np.array([1, 3])
	y = X[indices, :]
	"""
	[[ 6  7  8  9 10]
	[16 17 18 19 20]]
	"""
	z = X[:, indices]
	"""
	[[ 2  4]
	[ 7  9]
	[12 14]
	[17 19]]
	"""
	
	# NumPy built in slicing methods
	# ex: diagonals
	z = np.diag(X)
	# [ 1  7 13 19]
	# specify k to shift the diagonal indexing
	z = np.diag(X, k = 1)
	# [ 2  8 14 20]
	
	# finding uniques
	X = np.array([[1, 2, 3], [4, 5, 6], [1, 5, 6]])
	# [1 2 3 4 5 6]
\end{python}