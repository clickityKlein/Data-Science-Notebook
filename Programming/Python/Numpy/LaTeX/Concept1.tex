% Creating and Saving NumPy ndarrays

Not needing too much of an explanation, here's a beginning look at creating basic NumPy arrays.
\\\\
Covered in this section:
\begin{itemize}
	\item np.array: turn a list, or list of lists, into a NumPy array object
	\item shape: return the dimensions of an array object
	\item size: return the total number of elements in an array object
	\item save \& load: arrays can be saved for future use
\end{itemize}

\begin{python}
	x = np.array([1 ,2, 3, 4, 5])
	print(x)
	# [1 2 3 4 5]
	
	x.shape
	# (5,)
	
	Y = np.array([[1, 2, 3], [4, 5, 6], [7, 8, 9], [10, 11, 12]])
	print(Y)
	"""
	[[ 1  2  3]
	[ 4  5  6]
	[ 7  8  9]
	[10 11 12]]
	"""
	
	Y.shape
	# (4, 3)
	Y.size
	# 12
	
	# Notes Section:
	# dtype changes with mixed data, and will uptype (int + floats -> all floats) 
	# numpy arrays can be saved for later use
	
	# save
	np.save('my_array', x)
	# load
	y = np.load('my_array.npy')
\end{python}
