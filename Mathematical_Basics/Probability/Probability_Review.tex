\documentclass{article}
\usepackage{listings}
\usepackage[colorlinks=true,linkcolor=black,anchorcolor=black,citecolor=black,filecolor=black,menucolor=black,runcolor=black,urlcolor=black]{hyperref}
\usepackage{pythonhighlight}
\usepackage{graphicx}
\usepackage{amsmath}
\usepackage{mathtools}
\usepackage{nicematrix}
\usepackage{amssymb}
\setlength{\parindent}{0pt}

\begin{document}
	\title{Probability Review}
	\author{Klein \\ carlj.klein@gmail.com}
	\date{}
	\maketitle


\section{Cards}
\textbf{Question1:} \\
Inclusion-Exclusion ID
\\
Hint: P(A $\bigcup$ B) = ?
\\
\textbf{Answer1:} \\
P(A $\bigcup$ B) = P(A) + P(B) - P(AB)
\\\\


\textbf{Question2:} \\
Define Mutually Exclusive
\\
\textbf{Answer2:} \\
If AB = $\emptyset \rightarrowtail$ A and B are mutually exclusive 
\\\\

\textbf{Question3:} \\
Conditional Probability and Corollary
\\
\textbf{Answer3:} \\
Definition: P(E$\mid$F) = $\frac{P(EF)}{P(F)}$ 
\\
Corollary: P(EF) = P(E) * P(F$\mid$E)
\\\\


\textbf{Question4:} \\
Multiplication Rule
\\
Hint: extension of conditional probability: $\rightarrow P(E_1 *** E_n)$ = ?
\\
\textbf{Answer4:} \\
$P(E_1 *** E_n) = P(E_1) * P(E_2\mid E_1) * P(E_3\mid E_2 * E_1) *** P(E_n\mid E_{n-1} *** E_1)$
\\\\


\textbf{Question5:} \\
Law of Total Probability
\\
Hint: Given a mutually exclusive and exhaustive set A, P($A_1$) + ... + P($A_k$) = 1, what can be deduced about the probability of an event B occurring?
\\
\textbf{Answer5:} \\
P(B)\\
= P(B$A_1$) + ... + P(B$A_k$)\\
= P($A_1$) * P(B $\mid$ $A_1$) + ... + P($A_k$) * P(B $\mid$ $A_k$)\\
= $\sum_{i=1}^k P(A_i) * P(B \mid A_i)$
\\\\


\textbf{Question6:} \\
Bayes' Theorem 
\\
\textbf{Answer6:} \\
Given a mutually exclusive and exhaustive set A, P($A_1$) + ... + P($A_k$) = 1, then\\
$P(A_j \mid B) = $\\
$\frac{P(A_jB)}{P(B)} = $\\
$\frac{P(A_j) * P(B \mid A_j)}{\sum_{i=1}^k P(A_i) * P(B \mid A_i)}$



\end{document}