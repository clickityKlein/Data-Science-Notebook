% Concept2: Dots Products & Cross Products

We'll take another look into multiplication, more specifically multiplication between vectors, by further investigating properties of the dot product and introducing the cross product.

\begin{itemize}
	\item \nameref{concept2.1}
	\item \nameref{concept2.2}
	\item \nameref{concept2.3}
	\item \nameref{concept2.4}
	\item \nameref{concept2.5}
	\item \nameref{concept2.6}
%	\item \nameref{concept2.7}
%	\item \nameref{concept2.8}
%	\item \nameref{concept2.9}
%	\item \nameref{concept2.10}
\end{itemize}


\subsection{Dot Product vs. Cross Products}\label{concept2.1}

\begin{itemize}
	\item Dot Product: How much two vectors point in the same direction.
	\item Cross Product: How much two vectors point in opposite directions.
\end{itemize}


\subsection{Dot Products Between Vectors}\label{concept2.2}
Relationship with vector length: "The square of the length of a vector is equal to the vector dotted with itself."
\\

Recall the distance formula (magnitude of a vector): $\DistanceFormula$

Although we haven't explicitly defined the dot product for just vectors, if we follow the definition for dot products between matrices, the dot product for vectors produces a single value.
\\
\begin{equation}
	\vec{v} \cdot \vec{v}  = v_1 \cdot v_1 + . . . + v_n \cdot v_n = v_1^2 + . . . + v_n^2
	\longrightarrow
	\vec{v} \cdot \vec{v} = \left(\sqrt{v_1^2 + . . . + v_n^2}\right)^2 = \left\Vert \vec{v} \right\Vert
\end{equation}

A interesting relationship, but let's jump into some more generic properties of the dot product between vectors:
\begin{itemize}
	\item commutative: $\vec{u} \cdot \vec{v} = \vec{v} \cdot \vec{u}$
	\item distributive: $(\vec{u} \pm \vec{v}) \cdot \vec{w} = \vec{u} \cdot \vec{w} \pm \vec{v} \cdot \vec{w}$
	\item associate: $(c \cdot \vec{u}) \cdot \vec{v} = c(\vec{u} \cdot \vec{v})$
\end{itemize}

Once again, to hit home the general vector dot product equation: 
\begin{equation}
	\DotProductVectors
\end{equation}


\subsection{Cauchy-Schwarz Inequality}\label{concept2.3}
The absolute value of a dot product is always less than or equal to the product of their lengths.

\begin{equation}
	\CauchyShwarz
\end{equation}

The inequality is equivalent only when the vectors are collinear (i.e $\vec{u} = c \cdot \vec{v}$).
\\

Given that equivalence means collinearity, the C-S Inequality gives us another way to test for linear independence, since collinearity implies linear dependence.
\\


\subsection{Vector Triangle Inequality}\label{concept2.4}
The sum of the lengths of two sides of a triangle will always be greater than or equal to the length of the third side.
\\

Given sides of a triangle a, b, and c:
\begin{equation}
	c \leq a + b
\end{equation}

Written in terms of vectors:
\begin{equation}
	\VectorTriangle
\end{equation}

Once again, the inequality is equivalent only when the vectors are collinear and implies linear dependence.


\subsection{Angle Between Vectors}\label{concept2.5}
So far we've been preoccupied with magnitude, but what other relationships can we find between vectors. How about angle?
\\

Given the following relationship:
\begin{equation}
	\CosDotProduct
\end{equation}

What further information can solving for $\theta$ provide?
\\

One of the more crucial relationships is finding when vectors are perpendicular.
\begin{equation}
	\theta = 90^\circ \longrightarrow
	\cos (90^\circ) = 0 \longrightarrow
	\vec{u} \cdot \vec{v} = 0
\end{equation}

In other words, in the case neither of the vectors are the zero vector, a dot product between vectors resulting in 0 implies the vectors are perpendicular.
\\

Orthogonality: The concept of perpendicular vectors makes sense in 2-dimensions, but to extend to n-dimensions requires the idea of orthogonality.
\\

In general, $\vec{u} \cdot \vec{v} = 0$ (and provided neither of the vectors are the zero vector) implies the vectors are orthogonal.


\subsection{Equation of a Plane, and Normal Vectors}\label{concept2.6}