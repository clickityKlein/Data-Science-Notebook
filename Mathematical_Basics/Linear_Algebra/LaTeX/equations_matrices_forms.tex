% Equations and Matrix forms
\documentclass{article}
\usepackage{listings}
\usepackage[colorlinks=true,linkcolor=black,anchorcolor=black,citecolor=black,filecolor=black,menucolor=black,runcolor=black,urlcolor=black]{hyperref}
\usepackage{pythonhighlight}
\usepackage{graphicx}
\usepackage{amsmath}
\setlength{\parindent}{0pt}


\begin{document}
	
	
\section{Generic Matrices}

%<*generic_A>
\begin{equation}
	\begin{pmatrix}
		a_{1, 1} &\; .\; .\; . & a_{1, n} \\
		.        &\; .\; .\; . & .        \\
		.        &\; .\; .\; . & .        \\
		.        & \; .\; .\; . & .        \\
		a_{m, 1} & \; .\; .\; . & a_{m, n}
	\end{pmatrix}
\end{equation}
%</generic_A>

\newcommand{\generic_B}
{
	\begin{pmatrix}
		b_{1, 1} &\; .\; .\; . & b_{1, n} \\
		.        &\; .\; .\; . & .        \\
		.        &\; .\; .\; . & .        \\
		.        & \; .\; .\; . & .        \\
		b_{m, 1} & \; .\; .\; . & b_{m, n}
	\end{pmatrix}
}

\newcommand{\generic_C}
{
	\begin{pmatrix}
		c_{1, 1} &\; .\; .\; . & c_{1, n} \\
		.        &\; .\; .\; . & .        \\
		.        &\; .\; .\; . & .        \\
		.        & \; .\; .\; . & .        \\
		c_{m, 1} & \; .\; .\; . & c_{m, n}
	\end{pmatrix}
}


\end{document}