\documentclass{article}
\usepackage{listings}
\usepackage[colorlinks=true,linkcolor=black,anchorcolor=black,citecolor=black,filecolor=black,menucolor=black,runcolor=black,urlcolor=black]{hyperref}
\usepackage{pythonhighlight}
\usepackage{graphicx}
\setlength{\parindent}{0pt}

\begin{document}
	\title{Linear Algebra and Python Applications}
	\author{Klein \\ carlj.klein@gmail.com}
	\date{}
	\maketitle

\section{Learning Objectives}
These notes are to be used in reviewing the basics of linear algebra, and providing an analogue in programming the concepts in Python. We'll try to keep this relevant to the data science pathway, but some asides may be taken for pleasure in mathematics and necessity!
\\\\

\begin{enumerate}
	\item \nameref{sec:concept1}
	\item \nameref{sec:concept2}
	\item \nameref{sec:concept3}
\end{enumerate}

\section{Matrix \& Vector Basics}\label{sec:concept1}
% Concept1: Matrix & Vector Basics

We'll begin with the basics of linear algebra, which are rooted in a discussion about vectors and matrices.

\begin{itemize}
	\item \nameref{concept1.1}
	\item \nameref{concept1.2}
	\item \nameref{concept1.3}
	\item \nameref{concept1.4}
	\item \nameref{concept1.5}
	\item \nameref{concept1.6}
	\item \nameref{concept1.7}
	\item \nameref{concept1.8}
	\item \nameref{concept1.9}
	\item \nameref{concept1.10}
\end{itemize}

\subsection{Operations on a Single Matrix}\label{concept1.1}

\subsection{Matrix Addition \& Subtraction}\label{concept1.2}

\subsection{Matrix Multiplication (Dot Product)}\label{concept1.3}

\subsection{Vector Basics}\label{concept1.4}

\subsection{Standard Basis Vector (SBV)}\label{concept1.5}

\subsection{Span}\label{concept1.6}

\subsection{Linear Dependence \& Independence}\label{concept1.7}

\subsection{Linear Subspaces}\label{concept1.8}

\subsection{Spans as Subspaces}\label{concept1.9}

\subsection{Basis}\label{concept1.10}



\section{Dot Products \& Cross Products}\label{sec:concept2}

\section{Matrix-Vector Products}\label{sec:concept3}

\end{document}